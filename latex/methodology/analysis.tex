The security of a passwordless password manager is of paramount importance to
protect sensitive user infromation and prevent unauthorized access.
In this subsection, we conduct a comprehensive security analysis, identifying
  potential attack vectors and outlining the defensive measures that have
  been implemented to mitigate tehse risks.

The defensive measures include industry-standard security protocols such as
  HTTPS, JWT/JWE, Argon2 hashing, salt and pepper techniques, CORS
  configuration, and secure cookie management.
Additionally, different JWTs are used for registration, authentication, and
  password creation/retrieval to minimize the risk of exposure.
These defensive measures collectively aim to provide a robust and secure
  architecture for LessPM, safeguarding against various threats and ensuring
  the confidentiality, integrity, and availability of user data.
Each element in the security analysis contains a rating, as well as a status
on implemented and unimplemented.
\begin{enumerate}[label=$\blacktriangleright$]
  \item \textbf{HTTP} [Rating: 5 -- Status: Implemented]

  \textbf{Weakness/Attack Vector:}

  Any server running HTTP (Hypertext Transfer Protocol) is passing data
  between a server and a connecting client.
  With HTTP, this data is unencrypted.
  A malefactor can eavesdrop this data, performing a man-in-the-middle attack,
  replacing information, reading tokens, perform header injections, Etc.
  with this unencrypted traffic.

  \textbf{Defensive Mechanism:}

  Upgrading the HTTP connection to HTTPS ensured that communication between
  LessPM's client and server is encrypted through Transport Layer
  Security (TLS).
  It provides confidentiality and integrity of data transmitted over the
  network, making it more difficult for attackers to intercept or tamper with
  sensitive information.

  \item \textbf{JSON Web Tokens (JWT)} [Rating: 5 -- Status: Implemented]

  \textbf{Weakness/Attack Vector:}

  JWTs are signed with a secret specific to the server.
  However, if this secret is discovered or leaked, a malefactor could use this
  information to sign their own tokens and provide their own information.
  Since JWTs are signed and not encrypted, any malefactor who receives access
  to a token can decode the Base64 format and read the tokens information in
  clear text.

  \textbf{Defensive Mechanism:}

  Properly signining the tokens with a strong algorithm prevents information
  from being tampered with.
  An attacker can still read the information in clear-text but can not forge
  their own tokens.
  LessPM uses RSASSA-PSS with Sha-512 to sign and encode tokens, decoding
  them with the keypair when received.
  This ensures integrity and authenticity of the JWT, as the signature is
  verified by the server.
  RSASSA-PSS is a robust and secure signature algorithm that provides protection
  against various attacks, such as collision attacks and length extension
  attacks.
  To further impose a level of security, LessPM takes advantage of different
  tokens for registration, authentication, and authorization.

  \item \textbf{JSON Web Encryption (JWE} [Rating: 3 -- Status: Implemented]

  \textbf{Weakness/Attack Vector:}

  JSON Web Encryption are JWTs signed and encrypted using an algorithm for
  signing and encryption.
  However, if this encryption algorithm is not potent or weak (such as a
  Caesar Cipher), an attacker can break the encryption or even encrypt their own
  tokens.
  JWEs are also subject to scenarios such as lack of key rotation, insecure
  length of the encryption key and improper implementation.

  \textbf{Defensive Mechanism:}

  To combat these Weakness/Attack Vectores, LessPM implements encryption partly inspired by
  the JWE standard.
  LessPM encrypts the token using AES-256, a strong encryption scheme capable
  of $2^{256}$ different keys.
  This ensures that the data is confidentially kept and protected from
  unauthorized access.
  While inspired, it is important to note that JWEs have not been properly
  implemented in LessPM (See Section~\ref{sec:futurework}).

  \item \textbf{Hashing Keys} [Rating: 5 -- Status: Implemented]

  \textbf{Weakness/Attack Vector:}

  Hashing is a one-way process to convert information (such as passwords)
  into a fixed-size string of characters, typically a fixed-length hash value.
  However, the hashing process is susceptible to a malefactor using
  precomputed hash-values for large number of possible passwords, stored in a
  lookup table called a Rainbow table attack.
  Hashes are also vulnerable to dictionary attacks, and not being properly
  implemented without using a salt and preferable a pepper, or brute-force
  attacks.

  \textbf{Defensive Mechanism:}

  LessPM takes advantage of the latest within public hashing functionality,
  through Argon2 to hash the AES-256 key.
  Argon2 constructs a memory-hard and computationally expensive hash that
  provides protection against brute-force, rainbow table, dictionary and
  side-channel attacks by requiring a significant amount of computational
  resources and time to compute the hash.
  Finally, LessPM takes advantage of using both a random salt unique for each
  password and a pepper (stored in the source code) of 128-bit for both.
  This creates a higher threshold for a malefactor by requiring access to the
  database and the source code to be able to quickly compute a hash.
  The salt and pepper are added to the key of the AES-256

  \item \textbf{Mitigating Password Authentication} [Rating: 5 -- Status:
  Implemented]

  \textbf{Weakness/Attack Vector:}

  A password is something a secret that a user knows.
  Secrets belonging to an individual will always be potentially accessible to
  a malefactor.
  There are multiple vectors that can be used, such as a user being careless
  and writing the password down on a piece of paper or storing the secret in an
  unencrypted file.
  Further, a password is required to be stored somewhere for a server to
  authenticate the user, preferably with an email for recovery.
  This creates two new vectors for a malefactor to exploit, the persisted
  storage on the server or the recovery process.

  \textbf{Defensive Mechanism:}

  LessPM takes advantage of WebAuthn to avoid the usage of passwords to
  authenticate the user.
  The user is required to have two vectors for authentication; their
  authenticator device and the biometrics, which is the user itself.
  It is, however, important to mention that any authenticator device is
  vulnerable to being compromised.
  Should a device be compromised or infected with malicious software (Malware),
  it could be used to intercept or manipulate authentication requests.

  \item \textbf{Storing passwords in plaintext}
  [Rating: 5 -- Status: Implemented]

  \textbf{Weakness/Attack Vector:}

  Storing a password can be done in plaintext in a database protected by a
  password.
  However, should an attacker get access to the database, all passwords and
  their plaintext are compromised.
  Storing the passwords in plaintext makes the information accessible to any
  individual with access to the database, which causes a privacy risk.

  \textbf{Defensive Mechanism:}

  LessPM encrypt all passwords with one of the strongest public symmetric
  encryption processes, AES-256.
  Each key for each password is unique, and consists of the 192-bit CID from
  WebAuthn generated when the user registered (See
  Section~\ref{subsec:password-encryption}).
  This CID is unique to each service the user is registered with.
  Along the 192-bit CID, 128-bit is used from a salt and 128-bit used from a
  pepper.
  The salt is stored with the entry in the database.
  Using both salt and pepper enhances the security of the derived key.

  \item \textbf{
    Cross-Site Scripting,
    Cross-Site Request Forgery
    \& Unauthorized Access} [Rating: 5 -- Status: Implemented]

  \textbf{Weakness/Attack Vector:}

  Any HTTP server improperly configured is subject Cross-Site Scripting (XSS)
  and CSRF\@.
  XSS is a process where a malefactor could inject a malicious script from a
  different origin, and execute said script in the context of a user's browser.
  This could lead to a situation where a malefactor could get access tokens
  or cookies, or interact with the webpage a user is viewing through the
  script.
  CSRF is a type of security vulnerability where a malefactor tricks a user
  to unknowingly make unauthorized requests on a trusted website, such as a
  bank or similar.
  This can lead to actions being performed on a user's behalf unintentionally,
  without the user's consent, leading to unauthorized access.

  \textbf{Defensive Mechanism:}

  CORS is a measure that enforces a strict policy for which domains and services
  are permitted to access certain resources on the server.
  LessPM takes advantage of CORS by permitting the server itself and the
  client associated domain to access resources on the server.
  This is a preventive measure put in place to allow the communication
  between the client and the serve, even though they are running on different
  ports, and potentially different domains.
  This ensures that only authorized clients can access the server's resources,
  preventing unauthorized cross-origin requests and protecting against
  cross-site scripting (XSS) and cross-site request forgery (CSRF)\footnote{
    It would be beneficial to properly implement a CSRF token in submitted
    user data as well, which the server could validate when a user performs a
    request.
  } attacks.

  \item \textbf{Cookies} [Rating: 5 -- Status: Implemented]

  \textbf{Weakness/Attack Vector:}

  A cookie is vulnerable to many aspects of security.
  Improperly stored, a cookie can be accessed through JavaScript, sent to
  different domains, get hijacked, excessive expire time, or even poisoned.\footnote{
    Cookie Poisoning is the process where an attacker modifies the content of
    the cookie to inject malicious data or bypass security controls.
  }

  \textbf{Defensive Mechanism:}

  LessPM only uses one cookie, which contains the encrypted JWT\@.
  The cookie is protected through built-in browser-features such as
  restricted to the same origin that the cookie came from, and cannot be sent
  anywhere else, further preventing XSS of sensitive information\@.
  The cookie is expired after 15 minutes, which is extensive amount of time
  for a user to have authorized access to their passwords, before the need to
  reauthenticate their identity.
  Upon creation, LessPM makes sure that the cookie becomes set to secure.
  This prevents the cookie from being sent over an insecure HTTP connection,
  limiting it to HTTPS\@.
  Finally, the cookie is HttpOnly, so that the cookie can't be access through
  JavaScript, reducing the attack vector of SS further.

  \item \textbf{\texttt{Authorization} Header}
  [Rating: 3 -- Status: Unimplemented]
  
  \textbf{Weakness/Attack Vector:}

  The RFC7519~\cite{RFC7519} specifies the \texttt{Authorization} header as
  the appropriate place to append the token.
  This is of particular concern in LessPM as the HTTP framework we used requires
  to specify exposure of the \texttt{Authorization} header in order for
  JavaScript to read it.
  Exposing it to the client entails exposing it to the malefactors as well.

  \textbf{Defensive Mechanism:}

  Since exposing the \texttt{Authorization} header opens up an attack surface
  to malefactors, storing this value in a secured cookie similar to other
  cookies handled in LessPM would leave out an attack surface.
  This would further increase security in LessPM, as a malefactor would then
  not be able to read the token out of the Authorization header directly and
  act on behalf of the user.

  \item \textbf{Password Hashing} [Rating: 1 -- Status: Unimplemented]

  \textbf{Weakness/Attack Vector:}

  An unhashed password can have severe consequences.
  It leaves passwords vulnerable to unauthorized access, increasing the risk of
  security breaches and password reuse attacks.
  Insiders with access to password storage may misuse or abuse plaintext
  passwords, violating best practices for security and compliance requirements.
  Proper password hashing with strong cryptographic algorithms and salt is
  critical to protect user passwords, prevent unauthorized access, and ensure
  compliance with security standards.

  \textbf{Defensive Mechanism:}

  We emphasized and recognized that only constructing a hash
  for the AES key leaves the stored password exposed \textbf{iff} AES
  gets broken or have an unknown zero-day failure.
  Adding some form of hashing for the password before encrypting it would
  serve a minor benefit, in case of encryption breakage.
  As of this writing, we are unsure how this hash would be properly
  implemented, given that we have nothing to construct the hash for the
  password.
  A potential suggestion would be to apply similar methods to how the
  encryption was performed.

  \item \textbf{Hardcoded AES for JWT} [Rating: 3 -- Status: Unimplemented]

  \textbf{Weakness/Attack Vector:}

  Each JWT is encrypted with the same AES-256 key.
  This serves as a threat to the whole system, and was implemented this way
  due to the lack of time during the implementation phase.
  This leaves LessPM exposed in scenario that one key becomes broken and all
  the tokens passed between LessPM's client and server can be decrypted by a
  malefactor.

  \textbf{Defensive Mechanism:}

  LessPM could benefit from implementing some kind of key rotation that would
  construct a new key for each day or each user.
  In a perfect scenario, we would have applied the same techniques for the
  JWT token as for previously described for the password, using the CID, a
  salt, and a pepper constructed upon authenticating.

  \item \textbf{Encrypted Passkey} [Rating 1 - Status: Unimplmented]

  \textbf{Weakness/Attack Vector:}

  The public key part stored in LessPM is unencrypted and stored in plaintext.
  This leaves the key exposed to being read directly from a malefactor.
  The CID of the key is used for vital information such as serving as part of
  the key for the password encryption.
  However, seeing that this CID is useless without proper access to the code
  to see key-construction, and the database to get access to stored, encrypted
  passwords, this receives a lower rating.

  \textbf{Defensive Mechanism:}

  The Passkey could be encrypted with the same process used with passwords.
  This would require a user to present their AD every time the CID is
  necessary, which is already the case in LessPM\@.

  \item \textbf{Session Hijack Preventions} [Rating 1 - Status: Implemented]

  \textbf{Weakness/Attack Vector:}
  In a stateless protocol such as HTTP, measures are needed to be taken in
  order to maintain authentication and authorization (See
  Section~\ref{subsec:auth-and-auth}).
  These measures include a session or a token of authorization.
  In both cases, these tokens are needed to be stored somewhere in a client,
  such as a cookie or client-side variable.
  Protecting these storage areas is desired at the highest level, so to
  mitigate the risk of a session hijacking.

  \textbf{Defensive Mechanism:}
  In LessPM, we implemented cookies protected by various in-browser
  mechanisms, such as setting the cookie's \texttt{Secure}, \texttt{HttpOnly},
  \texttt{SameSite=Strict}, and \texttt{expiry} time.r
  All of these measure help LessPM to mitigate session hijacking through and
  ensuring user authenticity and authentication integrity in LessPM\@.
  As a side note, it would have been advantageous to further implement a
  mechanism surrounding the attachment of the connecting IP address to the
  JWT\@.
  This could serve as a first-line of defence in order to avoid exposure of
  unencrypted tokens.\footnote{
    This would not work on a mobile device connected to mobile data,
    seeing that the IP address switches between connections.
  }

  \item \textbf{Properly Implemented JWE} [Rating 1 - Status: Partially
  Implemented]

  \textbf{Weakness/Attack Vector:}
  Incorrect implementation of JWE can lead to vulnerabilities and attack
  vectors.
  We discovered fairly late during the development of LessPM that the
  standard implementation of JWT does not encrypt the claims~\cite{RFC7519}.
  In itself, a malefactor can decode the Base64-encoding of a JWT and then
  read the claim in plaintext.
  This prompted us to hastily create an inspired version of JWE~\cite{rfc7516
  }, which could encrypt the token.
  This can be exploited by a malefactor, seeing that the key to encrypt the
  token is hardcoded\footnote{
    It would still take $2^{256}$ attempts in the worst-case scenario for a
    malefactor.
  }

  \textbf{Defensive Mechanism:}

  LessPM should properly implement JWEs.
  Proper implementation, adherence to best practices, and staying updated with
  the latest standards and guidelines are crucial to mitigate the risks
  associated with incorrectly implemented JWE versions.
  Regular security audits and testing are essential for identifying and
  addressing vulnerabilities.
  
  \item \textbf{Multiple Authenticators} [Rating 2 - Status: Unimplemented]
  
  \textbf{Weakness/Attack Vector:}

  In its current implementation, LessPM supports only a single registered
  authenticators per username to maintain a focused security approach.
  During registration, LessPM checks the database for an existing username
  similar to the incoming one and aborts the Registration Ceremony if a match
  is found.
  However, not fully implementing support for multiple ADs in LessPM creates
  a case of single point of failure, where if an authenticator is lost,
  damaged, or compromised, a user might lose access and be locked out of
  their account.
  This also leads to reduced scalability and flexibility, where new and
  elaborated technologies might now be supported.
  
  \textbf{Defensive Mechanism:}
  WebAuthn permits users to have multiple authenticators.
  It then follows that LessPM should also support multiple authenticators.

\end{enumerate}