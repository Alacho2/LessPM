The Advanced Encryption Standard (AES) is a symmetric\footnote{
  Symmetric in this context refers to the key being the same to encrypt and
  decrypt.
} key encryption algorithm.
Since its inception in 1998, it has become the gold standard to encrypt various
information across applications~\cite{schneier2015applied,rijndael_book}, being adopted as the
successor of Data Encryption Standard (DES) by The National Institute of 
Standards and Technology (NIST) in 2001~\cite{nist_aes_winner}.
AES operates on a fixed-sizes units of data referred to as \texttt{blocks}~\cite{nistfips197blocks},
supporting keys of sizes 128-, 192-, and 256-bit~\cite{nistfips197intro}.
The design is based on a Substitution-Permutation Network (SPN) structure, which
combines substitution and permutation to provide a high level of security
through multiple rounds of processing~\cite{nistfips197specification}.
AES with 256-bit key-length (henceforth referred to as AES-256 in the rest
of the document), employs a 256-bit key and consists of 14 rounds of encryption,
offering an advanced level of security compared to its counterparts with shorter
key lengths and fewer rounds~\cite{nistfips197256}.
In each round of encryption, AES-256 undergoes four primary transformations:
\begin{figure}[htbp]
  \begin{itemize}
    \item \textbf{SubBytes} is a non-linear substitution step where each byte is replaced with
    another according to a predefined lookup table.
    \item \textbf{ShiftRows} is a transposition step where each row of the state
    is shifted cyclically a certain number of steps.
    \item \textbf{MixColumns} is a mixing operation that operates on the columns of the state,
    combining the four bytes in each column.
    \item \textbf{AddRoundedKey} combines the subkey with the state\protect\footnotemark
    ~using a bitwise exclusive OR (XOR) operation.
  \end{itemize}
  \caption{The steps AES takes when encrypting and decrypting data~\cite{nistfips197specification}.}
  \label{fig:aessteps}
\end{figure}
\footnotetext{The term \texttt{state} refers to an intermediate result that changes as
the algorithm progress through its phases.}
\newline
The larger key-size in AES-256 provides an exponential increase in number of
possible keys, making it significantly more resilient to brute-force attacks
and further solidifying its position as a robust encryption standard for
safe-guarding sensitive information.\footnote{
  The practical number of potential keys for an AES-256 implementation is
  $2^{256}$ possibilities. This gives us an approximation of $1.1579209 \times 10^{77}$
  options.
  The number is theoretical, as this is a worst-case scenario of options that an
  attacker has to go through in order to find the right key.
}

