The Advanced Encryption Standard (AES) is a symmetric\footnote{
  Symmetric in this context refers to the same key to encrypt and
  decrypt.
} key encryption algorithm.
Since its inception in 1998, it has become the gold standard for encrypting various
information across applications~\cite{schneier2015applied,rijndael_book}, being
adopted as the successor of the Data Encryption Standard (DES) by The National
Institute of Standards and Technology (NIST) in 2001~\cite{nist_aes_winner}.
AES operates on fixed-sizes units of data referred to as \texttt{blocks}~\cite{nistfips197blocks},
supporting keys of sizes 128-, 192-, and 256-bit~\cite{nistfips197intro}.
A Substitution-Permutation Network (SPN) structure forms the basis of the design,
which achieves a high level of security through multiple rounds of processing by
combining substitution and permutation~\cite{nistfips197specification}.
AES with 256-bit key length (hereafter referred to as AES-256 in the rest
of the report), employs a 256-bit key and consists of 14 rounds of encryption,
offering an advanced level of security compared to its counterparts with shorter
key lengths and fewer rounds~\cite{nistfips197256}.
In each round of encryption, AES-256 undergoes four primary transformations,
operating on a $4\times4$ block, as seen in Figure~\ref{fig:aessteps}.
\begin{figure}[htbp]
  \begin{itemize}
    \item \textbf{SubBytes} is a non-linear substitution step where each byte is
    replaced with another according to a predefined lookup table.
    \item \textbf{ShiftRows} cyclically shifts each row of the state over a
    certain number of steps.
    of the State over varying numbers of bytes while preserving their original
    values.
    \item \textbf{MixColumns} is a process that works on the columns of the
    state by combining the four bytes in each column through a mixing operation.
    \item \textbf{AddRoundedKey} involves combining a subkey with the state\protect\footnotemark
    ~by applying a bitwise XOR operation.
  \end{itemize}
  \caption{The steps AES takes when encrypting and decrypting data~\cite{nistfips197specification}.}
  \label{fig:aessteps}
\end{figure}
\footnotetext{The term \texttt{state} refers to an intermediate result that changes as
the algorithm progress through its phases.}
\newline
The larger key-size in AES-256 provides an exponential increase in the number of
possible keys, making it significantly more resilient to brute-force attacks
and further solidifying its position as a robust encryption standard for
safeguarding sensitive information.\footnote{
  The practical number of potential keys for an AES-256 implementation is
  $2^{256}$ possibilities. This gives us an approximation of $1.1579209 \times 10^{77}$
  options.
  The number is theoretical, as this is a worst-case scenario of options an
  attacker must go through to find the right key.
}

