Through implementing LessPM, we aimed to create a barebone implementation that
could serve as a reliable Minimal Viable Product (MVP).
However, we recognize that more work is needed to further enhance and compliment
the product.
The related topics to further improve LessPM are listed below and briefly
discussed as a way to highlight potential drawbacks of the current version.

    \begin{enumerate}[label=$\blacktriangleright$]
        \item \textbf{Authorization Headers}
        \item \textbf{AES Key Encryption for Password}
        \item \textbf{Hardcoded AES for JWT}
        \item \textbf{Encrypted Passkey}
        \item \textbf{Attaching connecting IP to JWT}
        \item \textbf{Properly implement JWE}
        \item \textbf{Multiple Authenticators}
        \newline In its current implementation, LessPM supports a single
        registered authenticator per username to maintain a focused security
        approach.
        During registration, the server checks the database for an existing
        username similar to the incoming one and aborts the registration
        ceremony if a match is found.
        While WebAuthn permits users to have multiple authenticators, limiting
        this feature in the initial iteration of LessPM helps ensure a more
        controlled security environment.
        As the system evolves, considering addition of support for multiple
        authenticators can be weighed against potential security risks and
        benefits.
    \end{enumerate}