\documentclass[twocolumn]{article}
\usepackage[margin=1in]{geometry}
\usepackage[english]{babel}
\usepackage[T1]{fontenc}
\usepackage{cuted}
\usepackage[hyphens]{url}
\usepackage{hyperref}
\usepackage{graphicx}
\usepackage{ifthen}
\usepackage{multicol}
\usepackage{listings}
\usepackage[
    backend=biber,
    style=alphabetic,
]{biblatex}
\usepackage{enumitem}
\usepackage{appendix}
\usepackage{titling}
\usepackage[labelfont=bf]{caption}
\usepackage{setspace}
\usepackage{draftwatermark}
\usepackage{todonotes}
\usepackage{amssymb}
\addbibresource{main.bib}
\SetWatermarkText{Draft}
\onehalfspacing
\interfootnotelinepenalty=10000

\newcommand{\stefan}{Stefán Ólafsson}
\newcommand{\jacky}{Jacky Mallet}
\newcommand{\rem}{T-701-REM4}
\newcommand{\secu}{T-742-CSDA}

\newcommand{\titleHeading}{CryptoChronicles}
\newcommand{\titleSubheading}{LessPM's Quest to a Passwordless Utopian Ecosystem}
\newcommand{\subjectName}{\secu}
\newcommand{\instructorName}{\jacky}

\lstset{
    basicstyle=\ttfamily,
    frame=single,
    frameround=tttt,
    breaklines=true
}

\title{\titleHeading}
\author{Håvard Nordlie Mathisen}
\date{January, 2023}

\iffalse

"Breaking the Chains: Unleashing LessPM's Passwordless Revolution"
"Pass-go Paradox: The Magical Mystery of LessPM's Passwordless Power"
"No More Hide-and-Seek: Discovering LessPM's Cryptographic Wonderland"
"Keyless Kingdom: A Journey into LessPM's Ultra-Secure Realm"
"The Fort Knox of Password Managers: Unlocking LessPM's Passwordless Potential"
"Less is More: The Art of Mastering Passwordless Security with LessPM"
"CryptoChronicles: The Adventures of LessPM's Passwordless Pioneers"

"Phantom Passcodes: LessPM's Invisible Path to Ironclad Security"
"Unraveling the Enigma: Exploring LessPM's Passwordless Universe"
"Now You See Me, Now You Don't: LessPM's Passwordless Sleight of Hand"
"Cryptic Conundrums: Solving the Passwordless Puzzle with LessPM"
"The Digital Decoder Ring: Unlocking the Secrets of LessPM's Passwordless Mastery"
"Mission: Impeccable – LessPM's Quest for Ultimate Passwordless Security"
"Pass-Free Paradise: The Utopian World of LessPM's Passwordless Ecosystem"

\fi

\newboolean{draftmode}
\setboolean{draftmode}{true}

\setlength{\columnsep}{25pt}
\setlength{\skip\footins}{20pt}
\begin{document}
    \begin{titlepage}
        \begin{center}
            \includegraphics[scale=0.4]{images/HR_logo_hringur_transparent}
            % \vspace*{1cm}

            \huge
            \textbf{\titleHeading}\\
            \Large
            \textbf{\titleSubheading}

            \vspace{1cm}
            \large
            \subjectName

            \vspace{0.5cm}
            \Large
            Håvard Nordlie Mathisen\\
            \large
            Reykjavik University\\
            havard22@ru.is

            % \vspace{1.5cm}

            \vfill

            \normalsize
            \textit{Instructor}\\
            \instructorName

            \vspace{0.8cm}

            Department of Computer Science\\
            Reykjavik University\\
            March, 2023\\

        \end{center}
    \end{titlepage}


    \iffalse

    Purpose:
    Executive Summary: The primary goal of an executive summary is to present a condensed version of the report's main
    findings, conclusions, and recommendations. It is targeted at decision-makers or executives who may not have the
    time to read the entire report. The executive summary aims to persuade the reader to take action based on the
    presented information.

    Content:
    Executive Summary: The executive summary typically includes background information, the problem or issue addressed,
    methodology, key findings, conclusions, and recommendations.
    It may also emphasize the report's significance or the implications of the findings.

    LEngth:
    Executive Summary: Executive summaries are usually longer than abstracts, ranging from a few paragraphs to a
    few pages, depending on the length and complexity of the report.
    The executive summary aims to provide enough detail to give a comprehensive understanding of the report's key
    points while still being concise.

    \fi


    \ifthenelse{\boolean{draftmode}}{
        \begin{strip}
            \tableofcontents
        \end{strip}
    }


    \iffalse

    Abstract: The purpose of an abstract is to provide a brief, comprehensive
    overview of the report's content, including its objectives, methodology,
    results, and conclusions.
    It is primarily intended for researchers, academics, or other professionals who want to quickly assess whether the
    report is relevant to their interests before deciding to read the entire document.

    Abstract: An abstract usually contains a brief introduction to the topic, the research question or problem,
    the methodology used, a summary of the results, and the main conclusions drawn from the study.
    It does not typically include recommendations or opinions.

    Abstract: Abstracts are typically shorter than executive summaries, usually limited to a single paragraph or around
    150--300 words.
    The abstract provides a high-level overview of the report's content and is meant to be as concise as possible.

    \fi

    \section*{Abstract}\label{sec:abstract}

    Test\cite{lol}

    \addtocontents{TOC}{\protect\setcounter{tocdepth}{1}}
    \section{Introduction}\label{sec:introduction}
    In today's digital landscape, utilizing various identifiers (such as usernames,
email addresses, or phone numbers) combined with passwords has become a
prevalent method for verifying an individual's identity and ensuring their
authorization to access restricted materials.

This is not a novel approach.
Polybius' \textit{The Histories}~\cite{perseus_tufts} contains the first
documented use of passwords, describing how the Romans employed
``\textit{watchwords}'' to verify identities within the military.
This provided a transparent, simple way to allow or deny entry to restricted
areas of authorized personnel only.
The story of secret writing (in this context referenced as cryptography) goes
back the past 3000 years~\cite{history_cryptography_cryptanalysis}, where the
need to protect and preserve privacy between two or more individuals blossomed.

Fernando J. Corbató is widely credited as the father of the first
computer password when he was responsible for the Compatible Time-Sharing
System (CTSS) in 1961 at MIT~\cite{levy1984hackers}.
The system had a \texttt{"LOGIN"} command, which, when the user followed it by
typing \texttt{"PASSWORD"}, had its printing mechanism turned off to offer
the applicant privacy while typing the password~\cite{ctss_programmers_guide}.
Given the long history of passwords and their importance, one could argue that
it was a natural and judicious step in the evolution of computer systems.

The study \textif{"The Memorability and Security of Passwords"}, conducted in
2004, provides insights into password creation strategies, including tips on
improving password entropy and methods for easy recall of passwords~\cite{
    yan2000password}.
With an emphasis on diversity in character selection, password length, and
avoiding dictionary words, the study suggests that acronym-based passwords offer
a delicate balance between memorability and security~\cite{yan2000password}.

However, as technology has advanced, the limitations of password-based
authentication have become increasingly apparent, leading to the development of
more sophisticated methods like Universal Authentication Framework
(UAF)~\cite{fido_uaf_overview} and WebAuthn~\cite{webauthn_level_2} through the
Fast IDentity Online Alliance \href{https://fidoalliance.org}{(FIDO)} and
The \href{https://www.w3.org}{World Wide Web Consortium} (W3C).

\newcommand{\assymetricCrypto}{\footnote{Asymmetric cryptography uses a
key-pair consisting of public and private keys. The public key encrypts data,
while the private key decrypts it. The keys are mathematically related, but
    deriving one from the other is infeasible, ensuring secure communication and
    data exchange.}}

This report explores the implementation of LessPM, a passwordless password
manager that leverages WebAuthn (Web Authentication), an open standard and
collaborative development effort between FIDO and W3C to provide a secure
authentication experience through biometric scanners on devices such as a
smartphone or a hardware authenticator device to provide a secure experience.
Free from the constraints of traditional passwords, while placing a strong
emphasis on security.
\todo[inline]{
    LessPM is designed with a multi-layer security approach to ensure
    confidentiality and integrity of user authentication and the passwords
    belonging to other services that user's are registered on.
    By utilizing WebAuthn's asymmetric
    cryptographic~\assymetricCrypto nature~\cite{webauthn-2-registering},
    LessPM authenticates users strongly through the help of authenticator
    devices such as a smartphone or hardware authentication device in possession
    of the user.
    When registering in LessPM, the user is prompted to use a biometric
    sensor on their device. The device constructs a keypair that the
    implemented WebAuthn server receives and stores the public key of the
    keypair before registering the user.
    This process can be seen in Figure~\ref{fig:registration}.

    When the user attempts to authenticate in the future, the user is again
    prompted to use the biometric sensors on their device. Upon initiating this
    process, the server issues a challenge which is signed by the private key on
    the authenticator device, yielding a signature back when a biometric
    sensor is successful in authenticating the authenticity of the user.
    The server can then verify this signature with the help of the public key
    stored int he previous step to authenticate the user.
    This process can be seen in Figure~\ref{fig:authentication}.

    The aforementioned described processes constructs an environment such that a
    malefactor is required to first access the authenticator device and then
    to bypass the device's biometric sensors security in order to authenticate
    as the user.

    Passwords are encrypted using AES-256, a widely recognized symmetric
    encryption algorithm~\cite{schneier2000secrets,rijndael_book}, with a
    Credential ID (CID) that is derived from the public key belonging to the
    keypair from WebAuthn, a 128-bit randomly generated salt, unique for each
    password, and a 128-bit pepper.
}
By examining recent advancements in authentication mechanisms and the related
innovative potential of WebAuthn, we hope to illuminate the prospects of a
passwordless future in digital security.

The findings in this report is not an attempt of getting rid of trivial
concepts such as session hijacking, nor is it an empirical study of user's
perception of the technology seen.

    \section{Background}\label{sec:background}
    
In today's world of rapidly evolving technology, it is essential to have a
strong foundation in the underlying concepts and protocols that drive modern
systems.
This background chapter aims to provide a comprehensive understanding of the key
technologies and principles that are relevant to our report.
By delving into these fundamental topics, we can better appreciate their
significance and application in the context of the implementation, design, and
architecture presented in the subsequent chapters.

% We explore literature related to hunduntuunuhunn

We will provide background information on topics such as WebAuthn
(Section~\ref{subsec:webauthn}), JSON Web Tokens (JWT\@.
Section~\ref{subsec:jsonwebtokens}), JSON Web Encryption (JWE\@.
Section~\ref{subsec:jsonwebtokens}), hashing through Argon2
(Section~\ref{subsec:hashing-through-argon2}), and Advanced Encryption Standard
(AES. Section~\ref{subsec:aes}).

% You should probably blabber some more here but we will do that when we have gotten further with the actual writing.

\subsection{WebAuthn}\label{subsec:webauthn}

\newcommand{\credIdentifier}{\footnote{
  There is a requirement to check whether the credential identifier, generated
  by the Authenticator device exsits on the server.
  For further discussions on this topic, see section \hyperref[sec:futurework]{Future work}}}


\subsection{Advanced Encryption Standard with 256-bit}\label{subsec:aes}
The Advanced Encryption Standard (AES) is a symmetric\footnote{
  Symmetric in this context refers to the same key to encrypt and
  decrypt.
} key encryption algorithm.
Since its inception in 1998, it has become the gold standard for encrypting various
information across applications~\cite{schneier2015applied,rijndael_book}, being
adopted as the successor of the Data Encryption Standard (DES) by The National
Institute of Standards and Technology (NIST) in 2001~\cite{nist_aes_winner}.
AES operates on fixed-sizes units of data referred to as \texttt{blocks}~\cite{nistfips197blocks},
supporting keys of sizes 128-, 192-, and 256-bit~\cite{nistfips197intro}.
A Substitution-Permutation Network (SPN) structure forms the basis of the design,
which achieves a high level of security through multiple rounds of processing by
combining substitution and permutation~\cite{nistfips197specification}.
AES with 256-bit key length (hereafter referred to as AES-256 in the rest
of the report), employs a 256-bit key and consists of 14 rounds of encryption,
offering an advanced level of security compared to its counterparts with shorter
key lengths and fewer rounds~\cite{nistfips197256}.
In each round of encryption, AES-256 undergoes four primary transformations,
operating on a $4\times4$ block, as seen in Figure~\ref{fig:aessteps}.
\begin{figure}[htbp]
  \begin{itemize}
    \item \textbf{SubBytes} is a non-linear substitution step where each byte is
    replaced with another according to a predefined lookup table.
    \item \textbf{ShiftRows} cyclically shifts each row of the state over a
    certain number of steps.
    of the State over varying numbers of bytes while preserving their original
    values.
    \item \textbf{MixColumns} is a process that works on the columns of the
    state by combining the four bytes in each column through a mixing operation.
    \item \textbf{AddRoundedKey} involves combining a subkey with the state\protect\footnotemark
    ~by applying a bitwise XOR operation.
  \end{itemize}
  \caption{The steps AES takes when encrypting and decrypting data~\cite{nistfips197specification}.}
  \label{fig:aessteps}
\end{figure}
\footnotetext{The term \texttt{state} refers to an intermediate result that changes as
the algorithm progress through its phases.}
\newline
The larger key-size in AES-256 provides an exponential increase in the number of
possible keys, making it significantly more resilient to brute-force attacks
and further solidifying its position as a robust encryption standard for
safeguarding sensitive information.\footnote{
  The practical number of potential keys for an AES-256 implementation is
  $2^{256}$ possibilities. This gives us an approximation of $1.1579209 \times 10^{77}$
  options.
  The number is theoretical, as this is a worst-case scenario of options an
  attacker must go through to find the right key.
}



\subsection{Hashing through Argon2}\label{subsec:hashing-through-argon2}
Argon2 is a key-derivation function developed by Alex Biryukov, Daniel Dinu,
and Dmitry Khovratovich~\cite{argon2specs}.
Argon2 aims to provide a highly customizable function
tailored to the needs of distinct contexts~\cite{argon2specs}.
Additionally, the design offers resistance to both time-memory trade-off and
side-channel attacks as a memory-hard function~\cite{argon2specs}.

The function fills large memory blocks with pseudorandom data derived from the
input parameters, such as the password and salt.\footnote{
  A salt is a randomly generated sequence of characters, unique to each instance
  that gets hashed. Argon2's intension is to have a 128-bit salt for all
  applications but this can be sliced in half, if storage is a
  concern~\cite{argon2specs}.
}
The algorithm then processes the blocks non-linearly for a specified number of
iterations~\cite{argon2specs}.

The function offers three configurations, depending on the environment where the
function will run and what the risk and threat models are:

\begin{figure}[htbp]
  \centering
  \begin{itemize}
    \item \textbf{Argon2d} is a faster configuration and uses data-depending
    memory access.
    This makes it suitable for cryptocurrencies and applications with little to
    no threat of side-channel timing attacks.\protect\footnotemark
    \item \textbf{Argon2i} uses data-independent memory access.
    This configuration is more suitable for password hashing and key-derivation
    functions.\protect\footnotemark
    ~This configuration is slower due to making more passes over the memory as
    the hashing progresses.
    \item \textbf{Argon2id} is a combination of the two, beginning with
    data-dependent memory access before transitioning to data-independent
    memory access after progressing halfway through the process.
  \end{itemize}
  \caption{The three configurations of Argon2~\cite{argon2specs}.}
  \label{fig:argon2conf}
\end{figure}

\footnotetext{
  Side-channel timing attacks analyze execution time variations in cryptographic
  systems to reveal confidential data, exploiting differences in time caused by
  varying inputs, branching conditions, or memory access patterns.
}
\footnotetext{
  Due to the nature of prioritizing security, LessPM uses the third
  configuration. This is expanded upon in Section~\ref{subsec:hashing-aes-key}.}

Argon2, as a memory-intensive hashing function, demands substantial
computational resources from attackers attempting dictionary attacks.\footnote{
  A dictionary attack is an approach where an attacker tries to find a hash by
  searching through a dictionary of pre-computed hashes or generating hashes
  based on a dictionary commonly used by individuals or businesses.
}
This characteristic significantly hampers the feasibility of cracking passwords
using such attacks.
The algorithm's customizability allows users to adjust its behaviour based on
memory, parallelism, and iterations, catering to specific security requirements
and performance needs.
As these configurations are crucial for computing the original hash, Argon2
provides robust resilience against brute-force and side-channel attacks.
The resulting enhanced security makes Argon2 suitable for password storage and
key-derivation in various applications and systems.

In 2015, Argon2 won the Password Hashing Competition~\cite{passwordhashing}.\footnote{
  NIST's competition to find an encryption algorithm inspired the Password
  Hashing Competition, but it took place without NIST's endorsement.
}


\subsection{JSON Web Token \& JSON Web Encryption}\label{subsec:json-web-tokens}\label{subsec:jsonwebtokens}

JSON Web Token (JWT) was introduced as part of RFC 7519 in 2015~\cite{RFC7519}.
It is a compact URL-safe string intended to transfer data between two entities.
They can be used as part of an authentication and authorization scheme in a web
service, application, or API\@.
The data in the string is intended as a payload and is referenced as a
\texttt{claim}~\cite{RFC7519}.

A JWT typically consist of three parts: header, payload, and signature.
The header and payload are serialized into JavaScript Object Notation (JSON)
and then encoded using a Base64Url encoding to ensure a URL-safe
format~\cite{RFC7519}.

The token contains an expiry timestamp, which, when decoded, is validated if
the timestamp is not passed at the time of decoding.
JWTs can be cryptographically signed using various algorithms like Hash-Based
Message Authentication (HMAC), Rivest-Shamir-Adleman (RSA), or
Elliptic-Curve Digital Signature Algorithm (ECDSA) ensuring the integrity and
authenticity of the token~\cite{RFC7519}.
This prevents unknown authorities from constructing or hijacking existing
tokens.\footnote{
  Hijacking a token could happen by a man-in-the-middle attack.
  This is done by a third-party individual listening and intercepting traffic
  in order to either read data or input their own in a client's request.
  This would allow an attacker to gain access to privileged information.
}
The URL-safe format of a JWT is often performed through a Base64 string, which
permits larger bits of data to be sent in a compressed, safe\footnote{
  Safe in this context should not be interchanged with secure.
  We reference safe as a way to transfer the data over the selected protocol,
  in most cases HTTP(S).
} format~\cite{RFC7519}.
JWTs are widely used for scenarios like single sign-on (SSO), user
authentication, and securing API endpoints by providing an efficient, stateless
mechanism for transmitting information about the user's identity, permissions,
and other relevant data~\cite{karande2018securingnode}.
% This section might be split up into two parts, dividing JWT and JWE


    \section{Literature Review}\label{sec:literature-review}
    Passwordless Authentication (PA) is a growing field of study within Computer
Science, as traditional authentication methods like passwords are increasingly
recognized as vulnerable to attacks such as phishing\footnote{
  Phishing is a form of attack where a hacker tries to leverage Social
  Engineering to act as a trusted entity to dupe a victim to give away
  credentials by opening an email, instant message, or text message, then
  signing into a spoofed website, seeming legitimate~\cite{ripa2021emergence}.
} and credential stuffing.\footnote{
  Credential stuffing refers to the practice of using automated tools to
  inject compromised or stolen username and password combinations into web login
  forms with the aim of gaining unauthorized access to user
  accounts~\cite{owasp-credential-stuffing}.
}
Passwords and sensitive information can also be a victim of successful
brute-force attacks~\cite{bonneau2012science} through data leakages by hacking
or purchasing information on the dark Web.

In accordance to NIST~\cite{NIST:SP:800-171r2, NISTSP800-63-3}, authentication
should consist of covering one the three following principles:

\begin{figure}[htbp]
  \begin{itemize}
    \item \textbf{Something you know}:
      This can be a password, an answer to a personal question, or a Personal
      Identification Number (PIN).
    \item \textbf{Something you have}:
      A device that contains some kind of token or cryptographically signed keys.
    \item \textbf{Something you are}:
      Biometrics of any sort or kind.
    Facial recognition, retina scan, fingerprint and similar.
  \end{itemize}
  \caption{The Three Principles of Password Security~\cite{schneier2000secrets, NIST:SP:800-171r2}.}
  \label{fig:secprinciples}

\end{figure}

There are many approaches to handle passwordless authentication, or a second
step to authenticate with a common password.\footnote{
  Often referenced as Two-Factor Authentication or Multi-Factor Authenticaiton.
}
In 2022, Parmar, Et al.\ published
\textit{A Comprehensive Study on Passwordless Authentication}~\cite{parmar2022},
containing the most common ways that PA can be performed, referencing several
interesting solutions, their advantages and drawbacks~\cite{parmar2022}.
The study discovered that PA commonly gets accepted as the most frictionless
system of authentication for User Interfaces (UI)~\cite{parmar2022}.
Biometrics was mentioned as one of the methods of authentication, concluding
that biometrics capture universal, human traits, encouraging differentiation
from one another~\cite{parmar2022}.
The same study brings up the caution surrounding the loss of authentication
device, and how fingerprints can be compromised~\cite{parmar2022}.\footnote{
  The security implication of using the core concept of FIDO2's WebAuthn is
  subject to storage in the system on Apple specific devices~\cite{appleSecureEnclave}.
  On an Android device, the implementation is up to the manufactorer of the
  device, where Samsung has implemented a Physically Unclonable
  Function~\cite{lee2021samsung}.
}

One promising approach is the use of the FIDO Alliance's collaborative work with
W3C to create WebAuthn.
WebAuthn is permitting users to authenticate through biometric information
stored on a device in the user's possession (i.e.\ phone, computer) or a
physical security key (i.e.\ YubiKey, Nitrokey, etc.)~\cite{webauthn_level_2}.

\textit{Passwordless VPN using FIDO2 Security Keys: Modern authentication
security for legacy VPN systems}~\cite{huseynov2022passwordless} utilized
WebAuthn to create a Web interface to create credentials and authenticate to use
a VPN~\cite{huseynov2022passwordless}.
The client required a user to authenticate through the procedure of WebAuthn
(see Section~\ref{subsec:webauthn}).
On a successful request, the Remote Authentication Dial-In User Service (RADIUS)
would create a temporary username and password, which would then be transfered
as a response to the end-user, permitting them to copy and paste it into the
necessary client, or construct a batch file which would establish the correct
connection~\cite{huseynov2022passwordless}.



    \section{Methodology}\label{sec:methodology}
    Exploring the development and implementation of LessPM, a passwordless password
manager, our focus will be on the key components, technologies, and steps
that form the system's development process.
A crucial part of the development and system is its security and robustness.
Our approach encompasses an explanation of the system architecture, the
technologies and tools utilized and the development process to create the
prototype.
By providing a comprehensive account of the system development process, we
aim to enable readers to understand the technical aspects of our work, as well
as to assess the validity and relevance of our findings.

\subsection{Environment}\label{subsec:environment}
To implement the system, we went to approach the development from a type-safety
environment.

\subsubsection{Server}
We chose Rust as the programming language for our backend which provided
significant benefits in terms of the software development environment.

Rust's emphasis on safety and performance allowed us to create a highly
secure and efficient environment for our implementation, without the risks
typically associated with memory-related
vulnerabilities~\cite{rivera2019preserving}.
The built-in memory management and focus on concurrency ensures that our
software runs smoothly, which is crucial~\cite{fischer1985impossibility}
when dealing with sensitive user data.

Additionally, Rust's surrounding ecosystem is rapidly
growing~\cite{librs-stats}, containing a vast library of high-quality
crates\footnote{
  \texttt{Crate} is the Rust-specific name for package or library.
}, which enables rapid development and easy integration of various
functionality.
Utilizing Rust's distinctive features, the passwordless password manager provides
enhanced security and reliability in the context of user authentication
~\cite{rivera2019preserving}, showcasing the benefits of utilizing a modern
programming language.

The backend is running an instance of an HTTPS server, serving as a wrapper
for the sensitive data.\footnote{
  During development, we took advantage of the Authorization header, as is
  specified with in RFC 7519.
  However, the framework that we used for the server requires to specifically
  expose the usage of \texttt{Authorization header} in order to access it in the
  client.
  See Section~\ref{sec:futurework} for further explanation.
}
The Chromium developers mandate this constraint to guarantee that the pertinent
API is invoked exclusively within a secure context~\cite{webdev2021credential}.
During development, the secure connection was performed through self-signed
certificate for \texttt{``localhost''} domain.

To serve as a persistent storage to maintain the passwords, user accounts,
and other related data, LessPM adopted MongoDB\@.
MongoDB allowed us to take use of their NoSQL architecture, permitting
storing Object-like structures~\cite{mongodb2021nosql}.

\subsubsection{Client}
An important part of the implementation is to create a viable client that can
function as a visual entry-point to the server.
For simplicity of the project, we chose React as a framework.
React is a JavaScript library for building user interfaces, offering a
efficient and flexible approach to web development.
We focused on following The Law of Demeter (LoD)~\cite{lieberherr1990assuring},
and using a least-knowledge principle.
This assures that the client only retains the necessary information needed to
function, having no knowledge of the server nor its implementation between each
performed request.

As is customery when developing a system containing authentication and
authorization, we are taking advantage of JWT/JWE
(Section~\ref{subsec:json-web-tokens})
to pertain some information, in order to authorize the client between
requested resources.
The Base64-encoded data that is passed between the server and client is
strongly encrypted through AES256 (Section~\ref{subsec:aes}).
We took advantage of React's ability to construct a single-page application with
no routing capabilities, avoiding the possibility of utilizing any kind of URL
tampering or manipulation\footnote{
  URL Manipulation is a technique performed by malefactors where a URL can be
  modified to request resources that should otherwise be inaccessbile to a user.
} to attempt privilege escalation.

\subsection{WebAuthn}\label{subsec:webauthn-methodology}
We used WebAuthn to perform the passwordless authentication.
As an open standard, WebAuthn aims to provide a key-based authentication
scheme in order to strongly authenticate users.
Keys are generated on a device in a user's possession and each time a
user signs up for a new service, a new key gets randomly generated on the
user's device (See Section~\ref{subsec:webauthn}).

WebAuthn is intuitively designed to attempt to mitigate phishing,
man-in-the-middle, and brute-force attacks~\cite{webauthn_level_2}.
By leveraging the increasing market of smartphones with
biometrics~\cite{statista-biometric-transactions}, WebAuthn becomes a natural
extension in terms of Authentication.

\subsubsection{Registration}
When a user tries to register in LessPM, the first thing we do is check
whether a user with a similar name exists.
We deny the registration if the user exists.
If there are no user with that name within the system, we generate a unique ID
using UUID V4, and starts the necessary Registration Ceremony.

We generate a JWE during this process, which is signed with RSASSA-PSS
using SHA-512 before getting encrypted with AES256 (See
Section~\ref{sec:futurework}).
This claim is then attached to the request's \texttt{Authorization} header,
along with the Creation options from WebAuthn (See
Section~\ref{subsec:webauthn}).
The \texttt{expiration} time for this claim is set to one minute to allow the
user some time to perform the authentication.\footnote{
  WebAuthn describes a timeout performed within the system. In our case, this
  timeout is one minute.
  However, we elected this extra approach to further secure the project, and
  JWEs need a timeout. See Section~\ref{subsec:json-web-tokens}.
}
The claim expires after this minute and the user will then have to restart the
process.
This was a decision we made to further emphasize security within LessPM\@.

When the response comes back to the user, LessPM uses the browser built-in
WebAuthn API to prompt the user for their authentication (See
Section~\ref{subsec:webauthn}).
At this point, it is entirely up to the user how they decide to perform the
authentication.
In our case, we used an Apple iPhone 13 Pro Max, a Samsung S21, and a Samsung
AX \todo[inline]{Ask Eva what kind of phone it is} to test authentication.\footnote{
  Other alternatives would include a YubiKey, NitroKey, etc.
}
We scanned the QR codes prompted through our phones, which initiates the
key-pair generation on the device after a face scan\footnote{
  There is a question of concern that facial recognistion can be bypassed
  through using a photography.
  Apple takes advantage of a built-in sensors to
  scan depth, colors, and a dot projector to create a 3D scan of a person's
  face.
  This prevents usage of a photography to perform authentication through
  their FaceID and TrueDepth technology~\cite{apple-support}.
}.
The public key then gets transmitted to the browser and sent to the Relying
Party through a new request.
Before the request reaches the Relying Party, there is an authentication
middleware, which checks for the JWE that got sent earlier and denies the
request with an \texttt{Unauthorized} status code if the request is not valid.
\footnote{
  Validity in this contex means not timed out, tampered with, or similar.
}
The user is then stored in the database if WebAuthn can validate the key and
metadata sent with teh request, and then the user gets stored in the database,
along with the UUID generated.

\subsubsection{Authentication}
Authentication is quite similar to the registration process.
The database gets checked upon an incoming request to the server.
The server immediately rejects the request if the user does not exist in the
database.

Further, the server attempts an Authentication Ceremony by collecting the
public key from database.
Upon validation\footnote{
  Validation in this context only means that the key stored in the database
  is a valid key.
} of the public key, the server creates a new JWE, which also receives an
\texttt{expiry} of one minute for the user to authenticate (See
Section~\ref{subsec:webauthn}), before sending that and the challenge in the
response.
We then prompt the user to authenticate with their original authenticator
that they used in the registration step, having the authenticator sign the
challenge before performing a new request to the server with the signed
challenge.
The user is considered authenticated if the Relying Party accepts the signed
challenge.



\todo[inline]{Make sure to mention the 15 minutes when you write this.}

\subsubsection{Password Creation \& Retrieval}

\subsection{Cors}\label{subsec:cors}
With the server and client running separately on different ports, Cross-Origin
Resource Sharing (CORS) needed to be configured correctly.

When a web page tries to access a resource hosted on another domain, browsers
perform an additional request to the server, called a \texttt{``preflight''}.
The preflight request is responsible for determining whether the actual
request that the web page is trying to make to the server is allowed.
This request is done through the \texttt{OPTIONS} method in HTTP, and contain
some information about the origin, accepted Content-Type, and similar of the
actual request.
The server response to this with what methods and headers are allowed,
denying the actual request from ever happening if the preflight is not
successful.

We constructed a CORS \texttt{layer}\footnote{
  A layer is referred to in this context as a wrapper around all other requests.
} which contained the two domains for the server and client, including
credentials\footnote{
  To pass the JWT token back and forth between the server
} and then permitting the two HTTP methods \texttt{POST} and \texttt{GET}.
We made sure the \texttt{Content-Type, Authorization}, and \texttt{Cookie}
headers are permitted.

Any other methods or headers should abort the request in the preflight.

\subsection{Cookie}\label{subsec:cookie}
JavaScript can access and manipulate \texttt{Cookies}~\cite{he2019malicious}.
We are utilizing the browser's local cookie storage to attempt secure
authentication between requests.\footnote{
  The cookie storage in a browser is subject to any vulnerabilities that can
  be performed on a SQLite database while having access to the computer where
  it is running.
}
Attempting a couple of strategies listed below, we aim to fortify the cookie
that LessPM sets in the browser, against a malefactor:
\begin{itemize}
  \item \textbf{Strict SameSite}:
  This ensures that the cookie is protected against Cross-Site Request
  Forgery (CSRF) and inaccessible to domains of other origins than the one
  where the cookie got sent from.
  \item \textbf{Expires}:
  Since the JWE is only good for 15 minutes after the user authenticated, the
  cookie gets a similar Time-to-Live (TTL) mechanism.
  \item \textbf{Secure}:
  Making sure that a cookie is only transmitted over a secure connection
  through HTTPS\@.
  This encrypts the data being sent back and forth between the client and
  the server, attempting to avoid eavesdroppers.
  \item \textbf{HttpOnly}:
  Setting HttpOnly tells the browser to make this cookie inaccessible through
  JavaScript.
  This is important to avoid session hijacking.
\end{itemize}

\subsection{Password Encryption}\label{subsec:password-encryption}
A password can be stored and hashed using a \texttt{salt} and in a typical
authentication scheme.
The user will provide their UID along with their password, this will get
collected from the database and checked with a random salt\footnote{
  Salting is the process of adding a randomly generated string consisting of
  arbitrary characters to the password before creating a hash~\cite{
    Kharod2015}.
} that was generated when the user registered.

This complicates the process compared to what is described above, since
the passwords should be a randomly generated string that is unknown to the
applicant.


\subsection{Hashing}\label{subsec:hashing}

\subsubsection{Hashing AES-key}\label{subsubsec:hashing-aes-key}


\subsection{JWT \& JWE}\label{subsec:jwt}



% Content served over HTTPS through a self-signed certificate
% Axum as a framework to serve HTTP-content
% Cors configuration X
% Preflight-mention to the client X
% Allow authorization and cookie X
% cookie stored in HttpOnly (inaccessible to JS). And expiry time which is the same as the JWT.
% JWT - you only have one minute to login after the request is performed. RSASSA-PSS using SHA-512
% Find some information about the JWT and use as a reference to talk about the transiton to JWE
% From JWT -> JWE
% - For security purposes, there should have been a new key for EACH JWT
% process (future work)
%
% There are 15 minutes to interact and do what you want after you're signed in.
% Password is copiable from the client but does disappear after 15 seconds.
% The server acts as a one source of truth, client is ONLY rendering and doesn't store anything
% (other than the cookie)

% From PBKDF2 (SHA256) to Argon2
% Argon configuration related to my system
% Security over usability (In this case)

% A longer text about WebAuthn, part of the FIDO-2 standard, a validator key that is UNIQUE across all registrations
% - makes it more difficult to track (suseptiable to email and the likes still).
% The key is stored in the database with the user

% Explain the encryption process in detail
% From generating a 416-bit HASH; 192 from the client validator, 96 from a random padding, 128 from the pepper.

% First, we generate a "key". This key consists of:
% - 192 bits from the client validator.
%   - The validator is different in size, tested on an iPhone 13 pro max, Samsung S21, MacBook Pro 16 inch Intel i9.
%   - Since the validator is different in size, we stop at 24 byte. If the validator is shorter, we PAD.
%   - So password A has the validator + a potential, necessary padding to reach 24 byte
%   - Password B has the validator + a potential, necessary padding to reach 24 byte.
%   - The padding is unique for each password and each validator.
% - 96 bits of randomly generated padding.
% - 128 bits from a pepper.

% After the key construction, we generate a salt of 96 bit. This salt gets mixed
% into the Argon2 hash.
% This constructs the hash, and the hash acts as the key for AES, so it is
% random.

% We use this HASH as the key for AES256_GCM to encrypt the data (password)
% encoding that result in Base64, which is
% then stored in a database along with:
% - the username the user selected
% - the random_padding (that was used to generate the key)
% - the uuid assigned to the user (used for retrieval)
% - the nonce (mixed in with AES256)
% - the key_padding (which is the necessary bits to reach 24 of the validator)
% - the salt (the one getting mixed in with the hashing

% The password itself gets encrypted with AES256, where the generated is the
% hash from the previous step
% and an appended nonce of 12 byte.

% A mention to the libraries that are being used, all of them profiled and
% highly scruitiniced with millions of downloads

% a proof of concept, that is the client
% All the client knows is a cookie, which is a JWT, signed with RSA512

    \section{Summary}\label{sec:conclusion}
    This report describes the details of implementing a passwordless password
manager, using WebAuthn to authenticate users.
The aim of the project was to create a secure and efficient solutions for
managing passwords that doesn't rely on the need for a traditional password.

We came up with an intuitive way to encrypt passwords, instead of the
traditional hash.
To achieve this, AES-256 was applied and Argon2 was used to construct a hash out
of a key supplied through WebAuthn's Credential ID and a randomly generated
salt for both key-derivation function and the AES-256 key, combined with a
pepper for the latter.

The backend where WebAuthn and the password manager was running consisted of a
HTTPS server, serving content through a self-signed certificate.
Because the server and client were independent of each other, we also
configured CORS\@.

To keep a user authenticated between requests, we used an encrypted version
of a JWT, inspired by the JWE~\cite{rfc7516}\@.
This was achieved by storing the token in a fortified cookie in the client,
being encrypted on the server before it was stored.

Despite its potential benefits, our report also highlights some limitations
and challenges associated with passwordless authentication.
These include the potential lack of adaptability by users and the
possibility of losing the authenticator device.

Our findings suggest that future studies should explore ways to address the
issue of device loss in the context of passwordless authentication.
In particular, research should focus on developing methods for securely
recovering access to accounts and data when a user’s primary authentication
device is lost or stolen.
This could include the use of backup authentication methods, such as biometric
verification or recovery codes, as well as the development of secure protocols
for remotely revoking access to lost devices.
By addressing this challenge, we can enhance the security and reliability of
passwordless authentication systems.


    \section{Future work}\label{sec:futurework}
    Through implementing LessPM, we aimed to create a barebone implementation that
could serve as a reliable Minimal Viable Product (MVP).
However, we recognize that more work is needed to further enhance and compliment
the product.
The related topics to further improve LessPM are listed below and briefly
discussed as a way to highlight potential drawbacks of the current version.

    \begin{enumerate}[label=$\blacktriangleright$]
        \item \textbf{Authorization Headers}
        \newline Even if~\cite{RFC7519} specifies the \texttt{Authorization}
        header as the appropriate place to append the header, in a system
        using WebAuthn it would be better to construct a cookie with similar
        settings as mentioned earlier in the report.
        This is of particular concern as the HTTP framework we used requires
        to specify exposure of the \texttt{Authorization} header in order for
        JavaScript to read it.
        Exposing it to the client entails exposing it to malefactors as well.
        \item \textbf{Hashing the stored password}
        \newline We emphasized and recognized that only constructing a hash
        for the AES key leaves the stored password exposed \textbf{iff} AES
        gets broken or have an unknown zero-day failure.
        Adding some form of hashing for the password before encrypting it would
        serve an exceptional benefit.
        As of thies writing, we are unsure how this hash would be properly
        implemented, given that we have nothing to construct the hash for the
        password.
        \item \textbf{Hardcoded AES for JWT}
        \newline Each JWT is encrypted with the same AES\@.
        This serves as a great treat to the whole system and was implemented
        this way due to the lack of time during the implementation phase.
        In a perfect scenario, we would have applied the same technique for
        the JWT token as for the passwords.
        \item \textbf{Encrypted Passkey}
        \newline The Passkey that gets stored on the user object could be
        beneficial
        to encrypt.
        While the Argon2 hash of the key serves as a great way to require an
        attacker to get access to the codebase as well as the database,
        the passkey could be encrypted with AES and the Credential ID serve
        as part of the key to decrypt it.
        \item \textbf{Attaching connecting IP to JWT}
        \newline We would have liked to attach the connecting IP address to the
        JWT\@.
        Seeing that a JWT token is exposed to the client and then a form of
        session hijacking, attaching an IP address to the token serves as a
        first-line defence in order to avoid exposure unencrypted
        tokens.\footnote{
            This would not work on a mobile device connected to mobile data,
            seeing that the IP address switches between connections.
        }
        \item \textbf{Properly implement JWE}
        \newline During development, we thought that JWTs were encrypted and
        not just signed.
        We discovered the JWEs~\cite{rfc7516} late in the process and these
        were new to us.
        Due to lack of time, we therefore took the shortcut of just
        implementing the encryption process of JWE, not the remaining metadata.
        In the future, we would like to properly implement these and follow
        the standard.
        \item \textbf{Multiple Authenticators}
        \newline In its current implementation, LessPM supports a single
        registered authenticator per username to maintain a focused security
        approach.
        During registration, the server checks the database for an existing
        username similar to the incoming one and aborts the registration
        ceremony if a match is found.
        While WebAuthn permits users to have multiple authenticators, limiting
        this feature in the initial iteration of LessPM helps ensure a more
        controlled security environment.
        As the system evolves, considering addition of support for multiple
        authenticators can be weighed against potential security risks and
        benefits.
        \item \textbf{Return Passwords with Authentication}
        \newline As part of the authentication process, we could make sure
        that the password list is returned with the last request to
        authenticate.
        This would further emphasize security by not exposing passwords in a
        separate end-point.
        On the same note, it would be good to handle the decryption part of
        the JWE better, seeing that a misconfigured cookie sent to this
        endpoint now returns an error.
        Yet another technical debt as a cause of lack of time.
        This is, however, not something that brings the system to a halt,
        rather an error that needs handling.
    \end{enumerate}

    \section{Acknowledgement}\label{sec:acknowledgement}
    The implementation of this project was only possible with the dedicated and
hardworking community surrounding Rust.

As a thank you, we would like to acknowledge the following cargos and their
hardworking developers:
\href{https://crates.io/crates/jsonwebtoken}{jsonwebtoken},
\href{https://crates.io/crates/ring}{ring},
\href{https://crates.io/crates/webauthn-rs}{webauthn-rs}.

While there are other libraries to mention, these are the most prominent ones
that serve as the basis of the project.
All the libraries are well-respected within the community, with the former two
having several million downloads and the latter well over 100.000.

\textbf{ChatGPT}
Throughout the report, the authors utilized ChatGPT to thoroughly reiterate
concepts used during development.


    \printbibliography

    \onecolumn
    \appendix
    \section{Appendix}\label{sec:appendix}
    \begin{figure*}[htbp]
\subsection{Base64 and ciphertext}\label{subsec:base64-and-ciphertext}
        \begin{lstlisting}[
          label={lst:base64-listing},
        ]
uAaQghfTS0jpMA1WaYozepQ4/TpN13Y6tlKSzXG0l+epVZK+vjvD8BwMIlWVxvTaV3mcFxL665qBNsFg//81hpU8I660lq/LAXsPcdfq2vr8YRa14+GH+Gtw6YlqLrDU3E8Rhb/IlAvJ8u5VN8pwV1SmBZLTwL7AyEWlp4GodSrX4NSl5grIFoVwRq7kXofVu1aUToD6KJcmo0X
BnwG0KWpEPUZFPd76KcA/QfXHnJHQsaR2jKWFJdRCpbtJAacAbssJk/bJjioo3AS1caEVZbNJctp9xgqVvgQJPyhmYtMLqdjq/SocUscTrLSPiR2X0g5sWByNIm6ses2SJ3dtMYYeOr7+qtVOcoX+U7w2+uLorawsCcXCmXRunEKd5jXiydwXZjzPoKHaT8hGwDB8CNSHxg/JXrezEJ5JKwq3Gio3xEyjD09Cfq5qLn9kENbcDZ/uRZK6+Swxcqg1DYhGngPJbbPkbOpqcXxdLutybYgdGpFN0zCQw3/LNbxYzOBeVhVsXM+GOVEAcKgYpnyCILwKKtFUyaRX2Q7IjdBbKP8NpG7RWZKFtRBVc4YVdWSIRpfek1/lFkq1JrvgK/6KjyyR+m6sb2RzUzxNO1V5uTkH2m8cBUwQBUqjiNEgVQDzTeaYIZqH0j2Is4cblRcCsjoFgX3Nvh4/OvgpgQ==
        \end{lstlisting}
        \begin{lstlisting}[label={lst:ciphertext.listing}, frame=single]
KH0
Vi3z8:Mv:RqU;"UWy뚁6`5<#{qakpj.OȔU7pWTEu*ԥ
pF^ջVN(&E)jD=FE=)?Aǜбv%BI�n	Ɏ*(qeIr}
	?(fb*RlX"nz͒'wm1:NrS6譬,	™tnB5f<ϠOF0|ԇ^I+
*7LOB~j.d
E,1r5
Fmljq|].rmM05X^Vl\φ9Q�p|
*TɤWȍ[(
nYUsudFޓ_J&+,nodsS<M;Uy9oLJ U�M!=:};?:)
        \end{lstlisting}
        \caption{The Base64 encoding and relevant ciphertext after AES-256.}
    \end{figure*}
\end{document}
