\documentclass{article}
\usepackage[margin=1.5in]{geometry}
\usepackage[english]{babel}
\usepackage[T1]{fontenc}
\usepackage[hyphens]{url}
\usepackage{hyperref}
\usepackage{graphicx}
\usepackage{listings}
\usepackage{biblatex}
\usepackage{enumitem}
\addbibresource{main.bib}
\usepackage{appendix}
\usepackage{titling}
\usepackage[labelfont=bf]{caption}
\usepackage{setspace}
\usepackage{draftwatermark}
\usepackage{todonotes}
\usepackage{amssymb}
\SetWatermarkText{Draft}
\onehalfspacing

\newcommand{\stefan}{Stefán Ólafsson}
\newcommand{\jacky}{Jacky Mallet}
\newcommand{\rem}{T-701-REM4}
\newcommand{\secu}{T-742-CSDA}

\newcommand{\titleHeading}{CryptoChronicles}
\newcommand{\titleSubheading}{LessPM's Quest to a Passwordless Utopian Ecosystem}
\newcommand{\subjectName}{\secu}
\newcommand{\instructorName}{\jacky}

\title{\titleHeading}
\author{Håvard Nordlie Mathisen}
\date{January, 2023}

\iffalse

"Breaking the Chains: Unleashing LessPM's Passwordless Revolution"
"Pass-go Paradox: The Magical Mystery of LessPM's Passwordless Power"
"No More Hide-and-Seek: Discovering LessPM's Cryptographic Wonderland"
"Keyless Kingdom: A Journey into LessPM's Ultra-Secure Realm"
"The Fort Knox of Password Managers: Unlocking LessPM's Passwordless Potential"
"Less is More: The Art of Mastering Passwordless Security with LessPM"
"CryptoChronicles: The Adventures of LessPM's Passwordless Pioneers"

"Phantom Passcodes: LessPM's Invisible Path to Ironclad Security"
"Unraveling the Enigma: Exploring LessPM's Passwordless Universe"
"Now You See Me, Now You Don't: LessPM's Passwordless Sleight of Hand"
"Cryptic Conundrums: Solving the Passwordless Puzzle with LessPM"
"The Digital Decoder Ring: Unlocking the Secrets of LessPM's Passwordless Mastery"
"Mission: Impeccable – LessPM's Quest for Ultimate Passwordless Security"
"Pass-Free Paradise: The Utopian World of LessPM's Passwordless Ecosystem"

\fi

\begin{document}

    \begin{titlepage}
        \begin{center}
            \includegraphics[scale=0.4]{images/HR_logo_hringur_transparent}
            % \vspace*{1cm}

            \huge
            \textbf{\titleHeading}\\
            \Large
            \textbf{\titleSubheading}

            \vspace{1cm}
            \large
            \subjectName

            \vspace{0.5cm}
            \Large
            Håvard Nordlie Mathisen

            % \vspace{1.5cm}

            \vfill

            \normalsize
            \textit{Instructor}\\
            \instructorName

            \vspace{0.8cm}

            Department of Computer Science\\
            Reykjavik University\\
            Match, 2023\\

        \end{center}
    \end{titlepage}


    \iffalse

    Purpose:
    Executive Summary: The primary goal of an executive summary is to present a condensed version of the report's main
    findings, conclusions, and recommendations. It is targeted at decision-makers or executives who may not have the
    time to read the entire report. The executive summary aims to persuade the reader to take action based on the
    presented information.

    Content:
    Executive Summary: The executive summary typically includes background information, the problem or issue addressed,
    methodology, key findings, conclusions, and recommendations.
    It may also emphasize the report's significance or the implications of the findings.

    LEngth:
    Executive Summary: Executive summaries are usually longer than abstracts, ranging from a few paragraphs to a
    few pages, depending on the length and complexity of the report.
    The executive summary aims to provide enough detail to give a comprehensive understanding of the report's key
    points while still being concise.

    \fi


    \section*{Executive Summary}

    Test\cite{lol}

    \iffalse

    Abstract: The purpose of an abstract is to provide a brief, comprehensive overview of the report's content,
    including its objectives, methodology, results, and conclusions.
    It is primarily intended for researchers, academics, or other professionals who want to quickly assess whether the
    report is relevant to their interests before deciding to read the entire document.

    Abstract: An abstract usually contains a brief introduction to the topic, the research question or problem,
    the methodology used, a summary of the results, and the main conclusions drawn from the study.
    It does not typically include recommendations or opinions.

    Abstract: Abstracts are typically shorter than executive summaries, usually limited to a single paragraph or around
    150--300 words.
    The abstract provides a high-level overview of the report's content and is meant to be as concise as possible.

    \fi

    \section*{Abstract}

    \section*{Introduction}
    Polybius' \textit{The Histories}~\cite{perseus_tufts} contains the first documented use of passwords, describing how
    the Romans employed ``\textit{watchwords}'' to verify identities within the military.
    This provided a transparent, simple way to allow or deny entry to restricted areas of authorized personnel only.
    The story of secret writing (in this context referenced as cryptography) goes back the past 3000 years~\cite{history_cryptography_cryptanalysis},
    where the need to protect and preserve privacy between two or more individuals blossomed.

    Fernando J. Corbató is widely credited as the all-father of the first computer password when he was responsible for
    the Compatible Time-Sharing System (CTSS) in 1961 at MIT~\cite{levy1984hackers}.
    The system had a \texttt{"LOGIN"} command, which, when the user followed it by typing \texttt{"PASSWORD"},
    had its printing mechanism turned off to offer the applicant privacy while typing the password~\cite{ctss_programmers_guide}.
    Given the long history of passwords and their importance, one could argue that it was a natural and judicious
    step in the evolution of computer systems.

    In today's digital landscape, utilizing various identifiers (such as usernames, email addresses, or phone numbers)
    combined with passwords has become a prevalent method for verifying an individual's identity and ensuring
    their authorization to access restricted materials.

    In 2004, a study titled \textit{"The Memorability and Security of Passwords"}~\cite{yan2000password} was conducted into
    advising users on the entropy of passwords and ways someone can use to remember a or multiple passwords.
    A typical standard for larger organizations with a form of password creation system is to emphasize the diversity of
    smaller, capitalized characters, length, and not be commonly referred to in a dictionary~\cite{yan2000password}.
    The study analyzed the effectiveness of different password-creation strategies, suggesting that acronym-based
    passwords offer a delicate balance between memorability and security.

    However, as technology has advanced, the limitations of password-based authentication have become increasingly
    apparent, leading to the development of more sophisticated methods like Universal Authentication Framework (UAF) and
    WebAuthn through the Fast IDentity Online (FIDO) Alliance.

    This report delves into the implementation of LessPM, a password manager that leverages WebAuthn to provide a secure
    authentication experience, free from the constraints of traditional passwords, while placing a strong emphasis on
    security.
    By examining recent advancements in authentication mechanisms and the related innovative potential of WebAuthn, we
    hope to illuminate the prospects of a passwordless future in digital security.


    % \hyperref[sec:futurework]{security}

    \section*{Methodology}
    % Content served over HTTPS through a self-signed certificate
    % Axum as a framework to serve HTTP-content
    % Cors configuration
    % Preflight-mention to the client
    % Allow authorization and cookie
    % cookie stored in HttpOnly (inaccessible to JS). And expiry time which is the same as the JWT.
    % JWT - you only have one minute to login after the request is performed. RSASSA-PSS using SHA-512
    % Find some information about the JWT and use as a reference to talk about the transiton to JWE
    % From JWT -> JWE
    % - For security purposes, there should have been a new key for EACH JWT process
    % There are 15 minutes to interact and do what you want after you're signed in.
    % Password is copiable from the client but does disappear after 15 seconds.
    % The server acts as a one source of truth, client is ONLY rendering and doesn't store anything
    % (other than the cookie)

    % From PBKDF2 (SHA256) to Argon2
    % Argon configuration related to my system
    % Security over usability (In this case)

    % A longer text about WebAuthn, part of the FIDO-2 standard, a validator key that is UNIQUE across all registrations
    % - makes it more difficult to track (suseptiable to email and the likes still).
    % The key is stored in the database with the user

    % Explain the encryption process in detail
    % From generating a 416-bit HASH; 192 from the client validator, 96 from a random padding, 128 from the pepper.

    % First, we generate a "key". This key consists of:
    % - 192 bits from the client validator.
    %   - The validator is different in size, tested on an iPhone 13 pro max, Samsung S21, MacBook Pro 16 inch Intel i9.
    %   - Since the validator is different in size, we stop at 24 byte. If the validator is shorter, we PAD.
    %   - So password A has the validator + a potential, necessary padding to reach 24 byte
    %   - Password B has the validator + a potential, necessary padding to reach 24 byte.
    %   - The padding is unique for each password and each validator.
    % - 96 bits of randomly generated padding.
    % - 128 bits from a pepper.

    % After the key construction, we generate a salt of 96 bit. This salt gets mixed into the Argon2 hash.
    % This constructs the hash, and the hash acts as the key for AES, so it is random.

    % We use this HASH as the key for AES256_GCM to encrypt the data (password) encoding that result in Base64, which is
    % then stored in a database along with:
    % - the username the user selected
    % - the random_padding (that was used to generate the key)
    % - the uuid assigned to the user (used for retrieval)
    % - the nonce (mixed in with AES256)
    % - the key_padding (which is the necessary bits to reach 24 of the validator)
    % - the salt (the one getting mixed in with the hashing

    % The password itself gets encrypted with AES256, where the generated is the hash from the previous step
    % and an appended nonce of 12 byte.

    % A mention to the libraries that are being used, all of them profiled and highly scruitiniced with
    % millions of downloads

    % a proof of concept, that is the client
    % All the client knows is a cookie, which is a JWT, encrypted with RSA512

    \section*{Conclusion}

    \section*{Future work}\label{sec:futurework}
    Through implementing LessPM, we aimed to create a barebone implementation that could serve as a reliable Minimal
    Viable Product (MVP). However, we recognize that more work is needed to further enhance and compliment the product.
    These topics are listed below and briefly discussed as a way to highlight drawbacks of the current version of LessPM.

    \begin{enumerate}[label=$\blacktriangleright$]
        \item \textbf{Authorization Headers}
    \end{enumerate}

    \printbibliography
\end{document}
