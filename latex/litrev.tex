Passwordless Authentication (PA) is a growing field of study within Computer
Science, as traditional authentication methods like passwords are increasingly
recognized as vulnerable to attacks such as phishing\footnote{
  Phishing is a form of attack where a hacker tries to leverage Social
  Engineering to act as a trusted entity to dupe a victim to give away
  credentials by opening an email, instant message, or text message, then
  signing into a spoofed website, seeming legitimate~\cite{ripa2021emergence}.
} and credential stuffing.\footnote{
  Credential stuffing refers to the practice of using automated tools to
  inject compromised or stolen username and password combinations into web login
  forms with the aim of gaining unauthorized access to user
  accounts~\cite{owasp-credential-stuffing}.
}
Passwords and sensitive information can also be a victim of successful
brute-force attacks~\cite{bonneau2012science} through data leakages by hacking
or purchasing information on the dark Web.

In accordance to NIST~\cite{NIST:SP:800-171r2, NISTSP800-63-3}, authentication
should consist of covering one the three following principles:

\begin{figure}[htbp]
  \begin{itemize}
    \item \textbf{Something you know}:
      This can be a password, an answer to a personal question, or a Personal
      Identification Number (PIN).
    \item \textbf{Something you have}:
      A device that contains some kind of token or cryptographically signed keys.
    \item \textbf{Something you are}:
      Biometrics of any sort or kind.
    Facial recognition, retina scan, fingerprint and similar.
  \end{itemize}
  \caption{The Three Principles of Password Security~\cite{schneier2000secrets, NIST:SP:800-171r2}.}
  \label{fig:secprinciples}
\end{figure}

There are many approaches to handle passwordless authentication, or a second
step to authenticate with a common password.\footnote{
  Often referenced as Two-Factor Authentication or Multi-Factor Authenticaiton.
}
In 2022, Parmar, Et al\cite{parmar2022}, described the most common
ways that PA can be performed, referencing several interesting solutions,
their advantages and drawbacks.
The study discovered that PA commonly gets accepted as the most frictionless
system of authentication for User Interfaces (UI)~\cite{parmar2022}.
Biometrics was mentioned as one of the methods of authentication, concluding
that it captures universal, human traits, encouraging differentiation
from one another~\cite{parmar2022}.
The same study brings up the caution surrounding the loss of authentication
device, and how fingerprints can be compromised~\cite{parmar2022}.\footnote{
  The security implication of using the core concept of FIDO2's WebAuthn is
  subject to storage in the system on Apple specific devices~\cite{appleSecureEnclave}.
  On an Android device, the implementation is up to the manufactorer of the
  device, where Samsung has implemented a Physically Unclonable
  Function~\cite{lee2021samsung}.
}

One promising approach is the use of the FIDO Alliance's collaborative work with
W3C to create WebAuthn.
WebAuthn is permitting users to authenticate through biometric information
stored on a device in the user's possession (i.e.\ phone, computer) or a
physical security key (i.e.\ YubiKey, Nitrokey, etc.)~\cite{webauthn_level_2}.

In ~\cite{huseynov2022passwordless} Web interface with WebAuthn was utilized to
create credentials that a user could use for a VPN\.
The client required a user to authenticate through the procedure of WebAuthn
(see Section~\ref{subsec:webauthn}).
On a successful request, the Remote Authentication Dial-In User Service (RADIUS)
creates a temporary username and password, which would then be transferred
as a response to the end-user, permitting them to copy and paste it into the
necessary client, or construct a batch file which would establish the correct
connection~\cite{huseynov2022passwordless}.
It was suggested create a solution for a VPN client which embedded some browser
components~\cite{huseynov2022passwordless}.

The process of using PA reduces the number of attack towards a server the
considerably, because the server no longer has any user authentication
secrets~\cite{gordin2021moving}.
Gordin, Et al.\ describes using WebAuthn to implement PA into an OpenStack
environment, providing increased security and bypassing the risk of leaked
passwords to be used on other services~\cite{gordin2021moving}.\footnote{
  See Section~\ref{subsec:webauthn} for further explanation as to how this
  works.\todo[inline]{Decide on this footnote.}
}
They, however, utilized a fingerprint as the primary biometrics, citing cost as
a primary factor to discourage the use of FIDO2~\cite{gordin2021moving}.\footnote{
  Authentication was provided through the Keystone environment on the OpenStack
  platform.
}

Statistia reported in 2020 that between 77--86\% of smartphones now have a
form of biometric scanner built into their device
\cite{statista-biometric-transactions}.
Gorin, Et al.\ continues by mentioning that some individuals have trouble
using their biometric scanner or getting it to work properly on their
device~\cite{gordin2021moving}, which can be a potential drawback for user
adaptability.
Thoroughly investigating the usability aspect of FIDO2,~\cite{
  ghrobany2020fido2} found that a significant amount of users asked

According to a study conducted by Lyastani et al.\ in 2020\cite{ghrobany2020fido2},
a significant portion of users found the usage of WebAuthn and Fido2 standard
to be easy and secure, but with some concerns about losing access to their
accounts or fear of their accounts being accessed by others.
The study utilized a fingerprint Yubikey for the authentication process.
Despite some reservations, the study found that overall, WebAuthn and Fido2
have good usability for passwordless authentication.
The authors suggested that more research is needed to explore ways to
further increase user trust and confidence in the technology, as well as to
address concerns about the potential loss of access to accounts.\footnote{
  We believe that these concerns mostly are raised due to the usage of a YubiKey,
  and that using a phone-based authenticator would reveal other results.
}


\iffalse

Our results show that users consider FIDO2 passwordless
authentication as more usable and more acceptable than the
traditional password-based authentication, but also that concerns remain that impede many users’ willingness to abandon
passwords. Most notably, the fear of losing the authenticator
is not only connected with account recovery but also with
an imminent illegal access to the account and the need for
revocation—-a subjective threat model by users that differs
from the objective risk assessment of FIDO2.

\fi
