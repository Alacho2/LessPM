Cross-Origin Resource Sharing (CORS) must be configured correctly when the
server and client are running separately on different ports.

When a web page tries to access a resource hosted on another domain, browsers
perform an additional request to the server, called a \texttt{``preflight''}.
The preflight request determines whether the
request that the web page is trying to make to the server is allowed.
This request is done through the \texttt{OPTIONS} method in HTTP and contains
some information about the origin, accepted Content-Type, and similar for the
actual request.

LessPM responds to this with what methods and headers are allowed, denying
the actual request from ever happening if the preflight is not successful.

We constructed a CORS \texttt{layer}\footnote{
  In this context, we referred to a layer as a wrapper around all other
  requests.
} which contained the
two domains for the server and client, permitted 
credentials\footnote{
  To pass the JWT token back and forth between the server
} and then permitted the two HTTP methods 
\texttt{POST} and \texttt{GET}.
We also ensured the \texttt{Content-Type, Authorization}, and \texttt{Cookie}
headers are permitted.

Any other methods or headers should abort the request in the preflight.

\subsection{Cookie}\label{subsec:cookie}
JavaScript can access and manipulate \texttt{Cookies}~\cite{he2019malicious}.
We utilized the browser's local cookie storage to attempt secure authentication
between requests.\footnote{
  The cookie storage in a browser is subject to any vulnerabilities that can be
  performed on an SQLite database while having access to the computer where it
  is running.
}
We attempted a couple of strategies listed below to fortify the cookie that
LessPM set in the browser against a malefactor:
\begin{itemize}
  \item \textbf{Strict SameSite}:
  This ensures that the cookie is safeguarded against Cross-Site Request Forgery
  (CSRF) attacks and remains restricted to its original origin domain.
  \item \textbf{Expires}:
  Once the system authenticated the user, it gave the cookie a Time-to-Live
  (TTL) mechanism similar to the JWE, which remained valid for only 15 minutes.
  \item \textbf{Secure}:
  We applied the \texttt{Secure} attribute to ensure that the cookie was only
  accessible through the HTTPS protocol.
  This protocol encrypts the data being sent back and forth between the client
  and the server, attempting to avoid eavesdroppers.
  \item \textbf{HttpOnly}:
  Setting HttpOnly tells the browser to make this cookie inaccessible through
  JavaScript.
  This property is important to mitigate session hijacking.
\end{itemize}