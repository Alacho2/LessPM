Through implementing LessPM, we aimed to create a barebone implementation that
could serve as a reliable Minimal Viable Product (MVP).
However, we recognize that more work is needed to further enhance and compliment
the product.

The related topics to further improve LessPM are listed below and briefly
discussed as a way to highlight potential drawbacks of the current version.

    \begin{enumerate}[label=$\blacktriangleright$]
        \item \textbf{Attaching connecting IP to JWT}
        \newline We would have liked to attach the connecting IP address to the
        JWT\@.
        Seeing that a JWT token is exposed to the client and then a form of
        session hijacking, attaching an IP address to the token serves as a
        first-line defence in order to avoid exposure of unencrypted
        tokens.\footnote{
            This would not work on a mobile device connected to mobile data,
            seeing that the IP address switches between connections.
        }
        \item \textbf{Properly implement JWE}
        \newline During development, we thought that JWTs were encrypted and
        not just signed.
        We discovered the JWEs~\cite{rfc7516} late in the process and these
        were new to us.
        Due to lack of time, we therefore took the shortcut of just
        implementing the encryption process of JWE, not the remaining metadata.
        In the future, we would like to properly implement these and follow
        the standard.
        \item \textbf{Multiple Authenticators}
        \newline In its current implementation, LessPM supports a single
        registered authenticator per username to maintain a focused security
        approach.
        During registration, the server checks the database for an existing
        username similar to the incoming one and aborts the registration
        ceremony if a match is found.

        While WebAuthn permits users to have multiple authenticators, limiting
        this feature in the initial iteration of LessPM helps ensure a more
        controlled security environment.
        As the system evolves, considering addition of support for multiple
        authenticators can be weighed against potential security risks and
        benefits.
        \item \textbf{Return Passwords with Authentication}
        \newline As part of the authentication process, we could make sure
        that the password list is returned with the last request to
        authenticate.
        This would further emphasize security by not exposing passwords in a
        separate end-point.

        On the same note, it would be good to handle the decryption part of
        the JWE better, seeing that a misconfigured cookie sent to this
        endpoint now returns an error.
        Yet another technical debt as a cause of lack of time.
        This is, however, not something that brings the system to a halt,
        rather an error that needs handling and nothing about what error it
        is exposed to the requesting party.
    \end{enumerate}