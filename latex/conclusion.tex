This report describes the details of implementing a passwordless password
manager, using WebAuthn to authenticate users.
The aim of the project was to create a secure and efficient solutions for
managing passwords that doesn't rely on the need for a traditional password.

We came up with an intuitive way to encrypt passwords, instead of the
traditional hash.
To achieve this, AES-256 was applied and Argon2 was used to construct a hash out
of a key supplied through WebAuthn's Credential ID and a randomly generated
salt for both key-derivation function and the AES-256 key, combined with a
pepper for the latter.

The backend where WebAuthn and the password manager was running consisted of a
HTTPS server, serving content through a self-signed certificate.
Because the server and client were independent of each other, we also
configured CORS\@.

To keep a user authenticated between requests, we used an encrypted version
of a JWT, inspired by the JWE~\cite{rfc7516}\@.
This was achieved by storing the token in a fortified cookie in the client,
being encrypted on the server before it was stored.

Despite its potential benefits, our report also highlights some limitations
and challenges associated with passwordless authentication.
These include the potential lack of adaptability by users and the
possibility of losing the authenticator device.

Our findings suggest that future studies should explore ways to address the
issue of device loss in the context of passwordless authentication.
In particular, research should focus on developing methods for securely
recovering access to accounts and data when a user’s primary authentication
device is lost or stolen.
This could include the use of backup authentication methods, such as biometric
verification or recovery codes, as well as the development of secure protocols
for remotely revoking access to lost devices.
By addressing this challenge, we can enhance the security and reliability of
passwordless authentication systems.
