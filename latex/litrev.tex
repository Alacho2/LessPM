Passwordless Authentication (PA) is a growing field of study within Computer
Science, as traditional authentication methods like passwords are increasingly
recognized as vulnerable to attacks such as phishing\footnote{
  Phishing is a form of attack where a hacker tries to leverage Social
  Engineering to act as a trusted entity to dupe a victim to give away
  credentials by opening an email, instant message, or text message, then
  signing into a spoofed website, seeming legitimate~\cite{ripa2021emergence}.
} and credential stuffing.\footnote{
  Credential stuffing refers to the practice of using automated tools to
  inject compromised or stolen username and password combinations into web login
  forms with the aim of gaining unauthorized access to user
  accounts~\cite{owasp-credential-stuffing}.
}
Passwords and sensitive information can also be a victim of successful
brute-force attacks~\cite{bonneau2012science} through data leakages by hacking
or purchasing information on the dark Web.

In accordance to NIST~\cite{NIST:SP:800-171r2, NISTSP800-63-3}, authentication
should consist of covering one the three following principles:

\begin{figure}[htbp]
  \begin{itemize}
    \item \textbf{Something you know}:
      This can be a password, an answer to a personal question, or a Personal
      Identification Number (PIN).
    \item \textbf{Something you have}:
      A device that contains some kind of token or cryptographically signed keys.
    \item \textbf{Something you are}:
      Biometrics of any sort or kind.
    Facial recognition, retina scan, fingerprint and similar.
  \end{itemize}
  \caption{The Three Principles of Password Security~\cite{schneier2000secrets, NIST:SP:800-171r2}.}
  \label{fig:secprinciples}

\end{figure}

There are many approaches to handle passwordless authentication, or a second
step to authenticate with a common password.\footnote{
  Often referenced as Two-Factor Authentication or Multi-Factor Authenticaiton.
}
In 2022, Parmar, Et al.\ published
\textit{A Comprehensive Study on Passwordless Authentication}~\cite{parmar2022},
containing the most common ways that PA can be performed, referencing several
interesting solutions, their advantages and drawbacks~\cite{parmar2022}.
The study discovered that PA commonly gets accepted as the most frictionless
system of authentication for User Interfaces (UI)~\cite{parmar2022}.
Biometrics was mentioned as one of the methods of authentication, concluding
that biometrics capture universal, human traits, encouraging differentiation
from one another~\cite{parmar2022}.
The same study brings up the caution surrounding the loss of authentication
device, and how fingerprints can be compromised~\cite{parmar2022}.\footnote{
  The security implication of using the core concept of FIDO2's WebAuthn is
  subject to storage in the system on Apple specific devices~\cite{appleSecureEnclave}.
  On an Android device, the implementation is up to the manufactorer of the
  device, where Samsung has implemented a Physically Unclonable
  Function~\cite{lee2021samsung}.
}

One promising approach is the use of the FIDO Alliance's collaborative work with
W3C to create WebAuthn.
WebAuthn is permitting users to authenticate through biometric information
stored on a device in the user's possession (i.e.\ phone, computer) or a
physical security key (i.e.\ YubiKey, Nitrokey, etc.)~\cite{webauthn_level_2}.

\textit{Passwordless VPN using FIDO2 Security Keys: Modern authentication
security for legacy VPN systems}~\cite{huseynov2022passwordless} utilized
WebAuthn to create a Web interface to create credentials and authenticate to use
a VPN~\cite{huseynov2022passwordless}.
The client required a user to authenticate through the procedure of WebAuthn
(see Section~\ref{subsec:webauthn}).
On a successful request, the Remote Authentication Dial-In User Service (RADIUS)
would create a temporary username and password, which would then be transfered
as a response to the end-user, permitting them to copy and paste it into the
necessary client, or construct a batch file which would establish the correct
connection~\cite{huseynov2022passwordless}.

