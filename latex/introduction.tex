In today's digital landscape, utilizing various identifiers (such as usernames, email addresses, or phone numbers)
combined with passwords has become a prevalent method for verifying an individual's identity and ensuring
their authorization to access restricted materials.

This is not a novel approach.
Polybius' \textit{The Histories}~\cite{perseus_tufts} contains the first documented use
of passwords, describing
how
the Romans employed ``\textit{watchwords}'' to verify identities within the military.
This provided a transparent, simple way to allow or deny entry to restricted areas of authorized personnel only.
The story of secret writing (in this context referenced as cryptography) goes
back the past 3000 years~\cite{history_cryptography_cryptanalysis}, where the
need to protect and preserve privacy between two or more individuals blossomed.

Fernando J. Corbató is widely credited as the all-father of the first
computer password when he was responsible for the Compatible Time-Sharing
System (CTSS) in 1961 at MIT~\cite{levy1984hackers}.
The system had a \texttt{"LOGIN"} command, which, when the user followed it by
typing \texttt{"PASSWORD"}, had its printing mechanism turned off to offer
the applicant privacy while typing the password~\cite{ctss_programmers_guide}.
Given the long history of passwords and their importance, one could argue that
it was a natural and judicious step in the evolution of computer systems.

The study \textif{"The Memorability and Security of Passwords"}, conducted in
2004, provides insights into password creation strategies, including tips on
improving password entropy and methods for easy recall of passwords~\cite{
    yan2000password}.
With an emphasis on diversity in character selection, password length, and
avoiding dictionary words, the study suggests that acronym-based passwords offer
a delicate balance between memorability and security~\cite{yan2000password}.

However, as technology has advanced, the limitations of password-based
authentication have become increasingly apparent, leading to the development of
more sophisticated methods like Universal Authentication Framework
(UAF)~\cite{fido_uaf_overview} and WebAuthn~\cite{webauthn_level_2} through the
Fast IDentity Online Alliance \href{https://fidoalliance.org}{(FIDO)} and
The \href{https://www.w3.org}{World Wide Web Consortium} (W3C).

\newcommand{\assymetricCrypto}{\footnote{Asymmetric cryptography uses a
key-pair consisting of public and private keys. The public key encrypts data,
while the private key decrypts it. The keys are mathematically related, but
    deriving one from the other is infeasible, ensuring secure communication and
    data exchange.}}
\newcommand{\jsonwebLibrary}{\footnote{
    \textbf{Note:} According to the library used to implement
    \href{https://docs.rs/jsonwebtoken/latest/jsonwebtoken/index.html}{jsonwebtokens in Rust}
    it is the private key that encrypts and the public key is responsible for \href{https://docs.rs/jsonwebtoken/latest/jsonwebtoken/struct.DecodingKey.html#method.from_rsa_pem}
    {decrypting}. \textit{Last Accessed: 2023-03-25}.}}

WebAuthn, short for Web Authentication, is an open standard for web-based
authentication that enables users to securely access online web services without
relying on a traditional password.
Through a collaborative effort between FIDO and W3C, WebAuthn is developed to
leverage asymmetric cryptography\assymetricCrypto\jsonwebLibrary~and biometric
or hardware-based authenticators to provide a more secure and robust
authentication experience.

This report explores the implementation of LessPM, a passwordless password
manager that leverages WebAuthn to provide a secure authentication experience,
free from the constraints of traditional passwords, while placing a strong
emphasis on security.
By examining recent advancements in authentication mechanisms and the related
innovative potential of WebAuthn, we hope to illuminate the prospects of a
passwordless future in digital security.
The findings in this report is not an attempt of getting rid of trivial
concepts such as session hijacking, nor is it an empirical study of user's
perception of the technology seen.