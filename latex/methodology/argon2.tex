Argon2 is a hasing/key-derivation function developed by Alex Biryukov, Daniel
Dinu, and Dmitry Khovratovich~\cite{argon2specs}.
Argon2 aims to provide a highly customizable function
tailored to the needs of distinct contexts~\cite{argon2specs}.
Additionally, the design offers resistance to both time-memory trade-off and
side-channel attacks as a memory-hard function~\cite{argon2specs}.

The function fills large memory blocks with pseudorandom data derived from the
input parameters, such as the password and salt.\footnote{
  A salt is a randomly generated sequence of characters, unique to each instance
  that gets hashed. Argon2's intension is to have a 128-bit salt for all
  applications but this can be sliced in half, if storage is a
  concern~\cite{argon2specs}.
}
The algorithm then processes the blocks non-linearly for a specified number of
iterations~\cite{argon2specs}.

The function offers three configurations, depending on the environment where the
function will run and what the risk and threat models are:

\begin{figure}[htbp]
  \centering
  \begin{itemize}
    \item \textbf{Argon2d} is a faster configuration and uses data-depending
    memory access.
    This makes it suitable for cryptocurrencies and applications with little to
    no threat of side-channel timing attacks.\protect\footnotemark
    \item \textbf{Argon2i} uses data-independent memory access.
    This configuration is more suitable for password hashing and key-derivation
    functions.\protect\footnotemark
    ~This configuration is slower due to making more passes over the memory as
    the hashing progresses.
    \item \textbf{Argon2id} is a combination of the two, beginning with
    data-dependent memory access before transitioning to data-independent
    memory access after progressing halfway through the process.
  \end{itemize}
  \caption{The three configurations of Argon2~\cite{argon2specs}.}
  \label{fig:argon2conf}
\end{figure}

\footnotetext{
  Side-channel timing attacks analyze execution time variations in cryptographic
  systems to reveal confidential data, exploiting differences in time caused by
  varying inputs, branching conditions, or memory access patterns.
}
\footnotetext{
  Due to the nature of prioritizing security, LessPM uses the third
  configuration. This is expanded upon in Section~\ref{subsec:hashing-aes-key}.}

Argon2, as a memory-intensive hashing function, demands substantial
computational resources from attackers attempting dictionary attacks.\footnote{
  A dictionary attack is an approach where an attacker tries to find a hash by
  searching through a dictionary of pre-computed hashes or generating hashes
  based on a dictionary commonly used by individuals or businesses.
}
This characteristic significantly hampers the feasibility of cracking passwords
using such attacks.
The algorithm's customizability allows users to adjust its behaviour based on
memory, parallelism, and iterations, catering to specific security requirements
and performance needs.
As these configurations are crucial for computing the original hash, Argon2
provides robust resilience against brute-force and side-channel attacks~\cite{
  argon2specs}.
The resulting enhanced security makes Argon2 suitable for password storage and
key-derivation in various applications and systems.

In 2015, Argon2 won the Password Hashing Competition~\cite{passwordhashing}.\footnote{
  NIST's competition to find an encryption algorithm inspired the Password
  Hashing Competition, but it took place without NIST's endorsement.
}
