
In today's world of rapidly evolving technology, it is essential to have a
strong foundation in the underlying concepts and protocols that drive modern
systems.
This background chapter aims to provide a comprehensive understanding of the key
technologies and principles that are relevant to our report.
By delving into these fundamental topics, we can better appreciate their
significance and application in the context of the implementation, design, and
architecture presented in the subsequent chapters.
We will provide background information on topics such as WebAuthn~\ref{subsec:webauthn},
JSON Web Tokens (JWT)~\ref{subsec:jsonwebtokens}, JSON Web Encryption (JWE)
\ref{subsec:jsonwebtokens}, hashing through Argon2~\ref{subsec:hashing-through-argon2},
and Advanced Encryption Standard (AES)~\ref{subsec:aes}.



% You should probably blabber some more here but we will do that when we have gotten further with the actual writing.

\newcommand{\credIdentifier}{\footnote{There is a requirement to check whether the credential identifier, generated by the Authenticator
device exsits on the server.
For further discussions on this topic, see section \hyperref[sec:futurework]{Future work}}}

\newcommand{\navigatorApi}{\footnote{In LessPM's case, this is the \textit{navigator.credentials} API provided by the browser.}}

\subsection{WebAuthn}\label{subsec:webauthn}
We used WebAuthn to perform passwordless authentication in LessPM\@.
WebAuthn is a collaborative, open standard between the FIDO Alliance and W3C\@.
The aim of the standard is to implement a secure, robust key-based
authentication system for the web, to strongly authenticate users through
biometrics~\cite{webauthn_level_2}.

The concept relies on the use of a third-party device, called an
Authenticator Device (AD), which leverages asymmetric cryptography.
These devices employ biometric or hardware-based mechanism to provide secure
and reliable means of authenticating a user in LessPM\@.

Upon registering in LessPM, the AD generated a key-pair called a Passkey.
This Passkey contained a CID uniquely generated for each registered
key-pair~\cite{webauthn_credential_id,webauthn_public_key_credential}, per
services registered on the AD\@.

Further, if a malefactor gains access to an individual's Passkey, they might
compromise one specific service, whereas a traditional password could
potentially compromise multiple services where password reuse
occurs~\cite{wang2018next}.
This suggests that WebAuthn could create a stronger level of security, whereas
a traditional password exposed in a leak might leave a user susceptible to
losing access across services and devices through the reused password.
Preventing the potential security threat of password reuse is a top priority
for LessPM\@.
That is why WebAuthn has been chosen as the preferred standard for
passwordless user authentication in LessPM\@.

WebAuthn further attempts to mitigate phishing, man-in-the-middle, and
brute-force attacks through its intuitive design~\cite{webauthn_level_2}.
By leveraging the increasing market of smartphones with
biometrics~\cite{statista-biometric-transactions}, WebAuthn becomes a natural
extension, not just for LessPM, but for general authentication.

In order to increase the security of authenticated requests, each of the
steps using WebAuthn, Registration~\ref{subsubsec:metho-registration},
Authentication~\ref{subsubsec:metho-authentication} and
Password- Creation \& Retrieval~\ref{subsubsec:creation-and-retrieval})receives
different tokens with different information and is constructed, verified, and
discarded separately in LessPM.\footnote{
  The token for password creation/retrieval is the same as verifying that a user
  is signed in and authorized to create entries on behalf of the user.
}
We made this decision to further elevate security in LessPM, so that an
exposed token could not serve as more than one entry point for one of the steps.

\subsubsection{Registration}\label{subsubsec:metho-registration}
When a user tried registering in LessPM, the client performs a
registration request to the server, called a Relying Party (RP), carrying a
relevant User Identifier (UID. i.e.\ username, phone number, email, Etc.).
This is done to check whether a user with a similar UID existed\footnote{
  WebAuthn does not require to check whether the CID generated by the AD exists.
  However, accepting users with similar identifier might leave a risk of
  providing unauthorized access.
  Hence why LessPM performs such a check.
} in the system, and LessPM denies the registration process if it does.

If there are no users with a similar UID, LessPM responds by initiating a
Registration Ceremony (RC), generated a challenge and a Unique User
Identifier (UUID), which serves as the body of the HTTPS response.
LessPM also attached a JWE to the HTTPS response \texttt{Authorization} header
which is sent to the client.

The \texttt{expiration} time for this claim is set to one minute to allow the
user some time to authenticate.\footnote{
  WebAuthn describes a timeout performed within the system. In our case, this
  timeout is one minute.
  However, we chose to add this extra step to secure LessPM further, and
  JWEs require an expiry time (See Section~\ref{subsec:auth-and-auth}).
}
The claim expired after this minute, and the user would then have to restart the
process.
This expiration timer was a decision we made to emphasize security within LessPM
further.

Using the issued challenge, LessPM's client called the browser-integrated
WebAuthn API\footnote{
  In LessPM's case, this is the \textit{navigator.credentials} API provided by
  the browser.
}, prompting the user to utilize their AD to create a new Passkey credential
through the Client To Authenticator Protocol (CTAP2)\footnote{
  The user is prompted to use their AD to prove their presence, which can
  involve facial recognition, providing a fingerprint, or any other modality
  supported by the device that the user chooses.
} for
LessPM\@.
At this point, it is entirely up to the user to decide what device to use to
authenticate.
For our development, we used an iPhone 13 Pro Max, a Samsung S21, and a
Samsung Galaxy A52 to test authentication.\footnote{
  Other alternatives included a YubiKey, NitroKey, Etc.
}
We scanned the QR codes prompted through our phones, which initiated the
key-pair generation on the AD after a successful biometric scan, such as
facial recognition.\footnote{
  There is a question of concern that photography can bypass facial recognition.
  Apple uses built-in sensors to scan depth, colours and a dot projector to
  create a 3D scan of a person's face.
  However, this approach prevents the use of photography to authenticate through
  their FaceID and TrueDepth technology~\cite{apple-support}.
}

Finally, the AD signs the challenge using the private key stored on the
device.
The created public key, signed challenge, and additional metadata are combined
into a public key credential object, which is forwarded to the client
through CTAP2 and then sent to LessPM in a new request.

Before the new request reaches the RP/LessPM, an authentication middleware
checks and validates the JWE, denying the request with an \texttt{Unauthorized}
HTTPS status code if the request is invalid.\footnote{
  Validity in this context means not timed out, tampered with, or similar.
}
If the RP/LessPM can validate the authenticity of the signed challenge and
public key through WebAuthn, the user was stored in the database, along with the
UUID generated.

Thus completing the RC\@.
This process can be seen in Figure~\ref{fig:webauthn}


\subsubsection{Authentication}\label{subsubsec:metho-authentication}
When a user wishes to perform authentication (commonly referred to as
\texttt{logging in}), much of the same procedure occurs in LessPM.
The user issued an authentication request in the LessPM's client, carrying
the UID the user used to register.
The client included this information in the body.

Upon receiving an incoming request, the RP/LessPM checked the database for
the containing UID\@.
LessPM immediately rejected the request, should the user not exist in the
database.
If LessPM found an associated user with the UID in the database, the server
responds by initiating and issuing an Authentication Ceremony (AC),
collecting the public key associated with the user from the database and
generating a new challenge.
Upon validation\footnote{
  Validation in this context only means that the key stored in the database
  is valid.
} of the public key, LessPM creates a new JWE, which, like registration, also
receives an \texttt{expiry} time of one minute for the user to authenticate,
attached to the HTTPS \texttt{Authorization} header.

The challenge is issued and LessPM's client then calls the
browser-integrated WebAuthn API again, prompting the use of the original AD
to validate and sign the challenge through CTAP2.
Unlike the registration process, the AD now yields a signed signature based
on the challenge issued by the RP/LessPM\@, which is transferred back to
LessPM's client through CTAP2 with some AD specific
data~\cite{webauthn_authenticator_data}.
A new HTTPS request is then issued with the signed challenge and the AD specific
data, and the RP/LessPM then validated the signature using the stored public
key.
LessPM considers the user authenticated if the RP accepted and validated the
signed challenge.
If the AC is unsuccessful at any point in the ceremony, the ceremony is
aborted and considered invalid.

Upon success a new JWE is generated.
This time, the \texttt{expiry} time-to-live is set to a 15-minute\footnote{
  The specification does not specify any upper- or lower bounds for the
  expiry time~\cite{RFC7519}
}
timeframe, allowing the user some time to perform necessary activities
within LessPM\@.
The WebAuthn-related process can be seen in Figure~\ref{fig:webauthn} and the
JWE-related process can be seen in Figure~\ref{fig:JWT-process}.

\subsubsection{Password Creation \& Retrieval}\label{subsubsec:creation-and-retrieval}
Passwords are sensitive in nature, so it seems only natural in a security
context to enforce an extra level of authentication upon retrieving and creating
one unique password in LessPM.\footnote{
  We retrieved a complete list of the user's passwords upon successful authentication.
  The hashed version of the password is stripped of the returned values to protect and enforce security.
}

The following options are presented to a user when they initiate a password
creation process in LessPM's client:

\begin{itemize}
  \item \textbf{User Identifier}: An identification that the user wants to
  associate with the password entry they are storing.
  Such as a username, phone number, or email.
  \item \textbf{Website}: A URL or similar where the password belongs.
  \item \textbf{Password}: The user is prompted with the input to create a
  strong password automatically, choosing options such as numbers, special
  symbols, smaller or larger characters, and the length.
  As an option, the user was also permitted to construct their password but
  warned by a warning saying that this option is less secure.
\end{itemize}

As a final step before a password is created and stored, the user is prompted to
reauthenticate with their AD\@.
This is done in a similar manner as described for registration (See
Section~\ref{subsubsec:metho-registration}) and authentication (See
Section~\ref{subsubsec:metho-authentication}).

To retrieve the plaintext version of the password, LessPM enforces a new
authentication request through the AD\@.
We made this decision to attempt to ensure the owner\footnote{
  In this context, we distinguish between \texttt{user} and \texttt{owner}
  as the person that created the password, not the person currently using
  LessPM.
} of
the password's presence\footnote{
  This approach is also used in password managers on phones to avoid situations
  where a user might have left their computer unlocked.
}.
In such a scenario where the owner left their computer open while logged in to
LessPM, a malefactor would not be able to simply retrieve a password at will.
To decrypt a password, LessPM also required the CID from the AD to perform
the key-reconstruction for the encryption algorithm employed in LessPM\@.
Further details about how encryption is achieved can be seen in
Section~\ref{subsec:password-encryption}.

\subsection{JSON Web Tokens \& JSON Web Encryption}\label{subsec:json-web-tokens}\label{subsec:jsonwebtokens}
Although JSON Web Tokens (JWTs)~\cite{RFC7519} are not inherently encrypted,
they still serve an essential purpose for some forms of secure data transfer.
By combining JWTs with JSON Web Encryption (JWE)~\cite{rfc7516}, secure data
transfer between a client and a server can be achieved.

\subsubsection{JSON Web Token}
JSON Web Token (JWT) was introduced as part of RFC 7519 in 2015~\cite{RFC7519}.
It is a compact URL-safe string intended to transfer data between two entities.
They can be used as part of an authentication and authorization scheme in a web
service, application, or API\@.
The data in the string is intended as a payload and is referenced as a
\texttt{claim}~\cite{RFC7519}.

A JWT typically consist of three parts: header, payload, and signature.
The header and payload are serialized into JavaScript Object Notation (JSON)
and then encoded using a Base64Url encoding to ensure a URL-safe
format~\cite{RFC7519}.

The token contains an expiry timestamp, which, when decoded, is validated if
the timestamp is not passed at the time of decoding.
JWTs can be cryptographically signed using various algorithms like Hash-Based
Message Authentication (HMAC), Rivest-Shamir-Adleman (RSA), or
Elliptic-Curve Digital Signature Algorithm (ECDSA) ensuring the integrity and
authenticity of the token~\cite{RFC7519}.
This prevents unknown authorities from constructing or hijacking existing
tokens.\footnote{
  Hijacking a token could happen by a man-in-the-middle attack.
  This is done by a third-party individual listening and intercepting traffic
  in order to either read data or input their own in a client's request.
  This would allow an attacker to gain access to privileged information.
}

The URL-safe format of a JWT is often performed through a Base64 encoding, which
permits larger bits of data to be sent in a compressed, safe\footnote{
  Safe in this context should not be interchanged with secure.
  We reference safe as a way to transfer the data over the selected protocol,
  in most cases HTTP(S).
} format~\cite{RFC7519}.
JWTs are widely used for scenarios like single sign-on (SSO), user
authentication, and securing API endpoints by providing an efficient, stateless
mechanism for transmitting information about the user's identity, permissions,
and other relevant data~\cite{karande2018securingnode}.

The process functions as follows:
\begin{itemize}
  \item The client completed an authentication request.
  \item A token is constructed and created through a claim on the server.
  \item The claim can be any data the server wishes to use to authenticate the
  legitimacy of a future request.
  \item The claim gets signed with a desired algorithm.
  This can also be a secret, stored on the server.
  \item As part of the response to a request, the server appends the token.\footnote{
    A common place to embed these tokens is in the Authroization part of the
    HTTP header~\cite{RFC7519}.
  }
  \item The client receives the token and carries it upon the next performed
  request.
  \item When the server receives the token again, it validates the \texttt{exp}
  property of the JSON object and takes action accordingly.
\end{itemize}

\subsubsection{JSON Web Encryption}
Complimenting the JWT standard are JWEs.
There are many parallels between the two but the major distinction between them
are that JWEs are encrypted~\cite{rfc7516}.
This encryption can happen through AES (see Section~\ref{subsec:aes}) and
provides integrity and authenticity for the token.
This prevents eavesdropping or tampering with the token during transit
% This section might be split up into two parts, dividing JWT and JWE

\subsection{Hashing through Argon2}\label{subsec:hashing-through-argon2}
Argon2 is a key-derivation function developed by Alex Biryukov, Daniel Dinu,
and Dmitry Khovratovich~\cite{argon2specs}.
Argon2 aims to provide a highly customizable function
tailored to the needs of distinct contexts~\cite{argon2specs}.
Additionally, the design offers resistance to both time-memory trade-off and
side-channel attacks as a memory-hard function~\cite{argon2specs}.

The function fills large memory blocks with pseudorandom data derived from the
input parameters, such as the password and salt.\footnote{
  A salt is a randomly generated sequence of characters, unique to each instance
  that gets hashed. Argon2's intension is to have a 128-bit salt for all
  applications but this can be sliced in half, if storage is a
  concern~\cite{argon2specs}.
}
The algorithm then processes the blocks non-linearly for a specified number of
iterations~\cite{argon2specs}.

The function offers three configurations, depending on the environment where the
function will run and what the risk and threat models are:

\begin{figure}[htbp]
  \centering
  \begin{itemize}
    \item \textbf{Argon2d} is a faster configuration and uses data-depending
    memory access.
    This makes it suitable for cryptocurrencies and applications with little to
    no threat of side-channel timing attacks.\protect\footnotemark
    \item \textbf{Argon2i} uses data-independent memory access.
    This configuration is more suitable for password hashing and key-derivation
    functions.\protect\footnotemark
    ~This configuration is slower due to making more passes over the memory as
    the hashing progresses.
    \item \textbf{Argon2id} is a combination of the two, beginning with
    data-dependent memory access before transitioning to data-independent
    memory access after progressing halfway through the process.
  \end{itemize}
  \caption{The three configurations of Argon2~\cite{argon2specs}.}
  \label{fig:argon2conf}
\end{figure}

\footnotetext{
  Side-channel timing attacks analyze execution time variations in cryptographic
  systems to reveal confidential data, exploiting differences in time caused by
  varying inputs, branching conditions, or memory access patterns.
}
\footnotetext{
  Due to the nature of prioritizing security, LessPM uses the third
  configuration. This is expanded upon in Section~\ref{subsec:hashing-aes-key}.}

Argon2, as a memory-intensive hashing function, demands substantial
computational resources from attackers attempting dictionary attacks.\footnote{
  A dictionary attack is an approach where an attacker tries to find a hash by
  searching through a dictionary of pre-computed hashes or generating hashes
  based on a dictionary commonly used by individuals or businesses.
}
This characteristic significantly hampers the feasibility of cracking passwords
using such attacks.
The algorithm's customizability allows users to adjust its behaviour based on
memory, parallelism, and iterations, catering to specific security requirements
and performance needs.
As these configurations are crucial for computing the original hash, Argon2
provides robust resilience against brute-force and side-channel attacks~\cite{
  argon2specs}.
The resulting enhanced security makes Argon2 suitable for password storage and
key-derivation in various applications and systems.

In 2015, Argon2 won the Password Hashing Competition~\cite{passwordhashing}.\footnote{
  NIST's competition to find an encryption algorithm inspired the Password
  Hashing Competition, but it took place without NIST's endorsement.
}


\subsection{Advanced Encryption Standard with 256-bit}\label{subsec:aes}
The Advanced Encryption Standard (AES) is a symmetric\footnote{
  Symmetric in this context refers to the key being the same to encrypt and
  decrypt.
} key encryption algorithm.
Since its inception in 1998, it has become the gold standard to encrypt various
information across applications~\cite{schneier2015applied,rijndael_book}, being adopted as the
successor of Data Encryption Standard (DES) by The National Institute of 
Standards and Technology (NIST) in 2001~\cite{nist_aes_winner}.
AES operates on a fixed-sizes units of data referred to as \texttt{blocks}~\cite{nistfips197blocks},
supporting keys of sizes 128-, 192-, and 256-bit~\cite{nistfips197intro}.
The design is based on a Substitution-Permutation Network (SPN) structure, which
combines substitution and permutation to provide a high level of security
through multiple rounds of processing~\cite{nistfips197specification}.
AES with 256-bit key-length (henceforth referred to as AES-256 in the rest
of the document), employs a 256-bit key and consists of 14 rounds of encryption,
offering an advanced level of security compared to its counterparts with shorter
key lengths and fewer rounds~\cite{nistfips197256}.
In each round of encryption, AES-256 undergoes four primary transformations:
\begin{figure}[htbp]
  \begin{itemize}
    \item \textbf{SubBytes} is a non-linear substitution step where each byte is replaced with
    another according to a predefined lookup table.
    \item \textbf{ShiftRows} is a transposition step where each row of the state
    is shifted cyclically a certain number of steps.
    \item \textbf{MixColumns} is a mixing operation that operates on the columns of the state,
    combining the four bytes in each column.
    \item \textbf{AddRoundedKey} combines the subkey with the state\protect\footnotemark
    ~using a bitwise exclusive OR (XOR) operation.
  \end{itemize}
  \caption{The steps AES takes when encrypting and decrypting data~\cite{nistfips197specification}.}
  \label{fig:aessteps}
\end{figure}
\footnotetext{The term \texttt{state} refers to an intermediate result that changes as
the algorithm progress through its phases.}
\newline
The larger key-size in AES-256 provides an exponential increase in number of
possible keys, making it significantly more resilient to brute-force attacks
and further solidifying its position as a robust encryption standard for
safe-guarding sensitive information.\footnote{
  The practical number of potential keys for an AES-256 implementation is
  $2^{256}$ possibilities. This gives us an approximation of $1.1579209 \times 10^{77}$
  options.
  The number is theoretical, as this is a worst-case scenario of options that an
  attacker has to go through in order to find the right key.
}

