
Exploring the development and implementation of LessPM, our focus will be on the
key components, technologies, and steps that form LessPM's development process
and encapsulation.
A crucial part of the development and system was its security and robustness.

Our approach encompasses an explanation of the system architecture, the
technologies and tools utilized and the development process to create the
prototype.
By providing a comprehensive account of the system development process, we aim
to enable readers to understand the technical aspects of our work as well
as to assess the validity and relevance of our findings.

\subsection{Environment}\label{subsec:environment}
To implement the system, we approached the development from a type-safety
environment.

\subsubsection{Server}
We chose Rust as the programming language for our backend, which provided
significant benefits regarding the software development environment.

Rust's emphasis on safety and performance allowed us to create a highly
secure and efficient environment for our implementation without the risks
typically associated with memory-related
vulnerabilities~\cite{rivera2019preserving}.
The built-in memory management and focus on concurrency ensures that the
software runs smoothly, which is crucial~\cite{fischer1985impossibility} when
dealing with sensitive user data.

Additionally, Rust's surrounding ecosystem is at rapid
growth~\cite{librs-stats}, containing a vast library of high-quality
crates\footnote{
  \texttt{Crate} is the Rust-specific name for a package or library.
}, which enables expeditious development and easy integration of various
functionality.
Utilizing Rust's distinctive features, the LessPM achieves enhanced security and
reliability in the context of user authentication~\cite{rivera2019preserving},
showcasing the benefits of utilizing a modern programming language.

LessPM ran an instance of an HTTPS server, serving as a wrapper for the
sensitive data.\footnote{
  We took advantage of the Authorization header during development, as specified
  in RFC 7519.
  However, the framework we used for the server required us to expose the usage
  of \texttt{Authorization header} to access it in the client.
  See Section~\ref{subsec:security-analysis} for further explanation.
}
The Chromium developers mandate this constraint to guarantee that the pertinent
API is invoked exclusively within a secure context~\cite{webdev2021credential}.
During development, we established a secure connection by using a self-signed
certificate for the \texttt{``localhost''} domain.
As a deliberate decision, we opted for MongoDB due to its NoSQL architecture,
which facilitated the storage of Object-like structures~\cite{mongodb2021nosql}
including passwords, user accounts, and other related data that LessPM stores.

\subsubsection{Client}
An essential part of the implementation was to create a viable client that could
function as a visual entry point to LessPM and proof-of-concept.
For the simplicity of the project, we chose React as a framework.
React is a JavaScript library for building user interfaces, offering an
efficient and flexible approach to web development.
The design of LessPM was influenced by the principles of
least-knowledge~\cite{lieberherr1990assuring}.
This influenced the decision to ensure that the client only retained
necessary information to function properly, containing no knowledge of
the server nor its implementation.

We took advantage of React's ability to construct a single-page application with
no routing capabilities, avoiding the possibility of utilizing any URL tampering
or manipulation\footnote{
  Malefactors can perform URL Manipulation, which involves modifying a URL to
  request resources that would otherwise be inaccessible to a user.
} to attempt privilege escalation or
accessing restricted data.

When initiating the project's development, we had a strong vision of creating
a client that could seamlessly integrate with Chromium-based browsers as
an extension.\footnote{
  In the client's project folder are traces of a manifest.json file and build
  scripts to have the client run in an extension.
}
However, we quickly discovered that the Rust cargo used to perform WebAuthn
requests did not implement this functionality.
We reported this issue
\href{https://github.com/kanidm/webauthn-rs/issues/288}{upstream to the authors}
of the cargo, and we will continue to work with the authors to ensure a proper
implementation of this functionality in the future.

LessPM's client maintains authentication and authorization through an
encrypted JSON Web Token (JWT, RFC7519), inspired by JSON Web
Encryption (JWE, RFC7516), passed back and forth between the server and client.

\subsection{Authentication \& Authorization}\label{subsec:jwt}
A system that maintains authentication and authorization through a stateless
protocol such as HTTPS, requires some information to authorize a
client between requested resources.
LessPM took advantage of JWTs and an inspired variation of JWEs for this
purpose.
Although JWTs are not inherently encrypted~\cite{RFC7519}, combined with JWE,
they serve as an essential backbone for secure data transfer in LessPM\@.

\subsubsection{JSON Web Tokens}
JWT is a compact, URL-safe string intended to transfer data between two
entities, often represented as a Base64 encoding.
They are often used as part of authentication and authorization scheme within
a web service, application, or Application Programming Interface (API), and
LessPM does this as well.
The data in the string is delivered as a payload and is referred to as a
\texttt{claim}~\cite{RFC7519}.

The JWT typically consist of three parts: header, payload, and signature.
The header and payload are serialized into JavaScript Object Notation (JSON)
and then encoded using Base64 to ensure a URL-safe format.
This permits larger bits of data to be sent in a compressed, safe\footnote{
  Safe in this context should not be interchanged with secure.
  We reference safe as a way to transfer the data over the selected protocol,
  in LessPM's case this HTTPS.
} format.
JWTs are widely used for scenarios such as single sign-on (SSO), user
authentication, and securing API endpoints by providing an efficient and
stateless mechanism for transmitting information about the user's identity
and permissions~\cite{karande2018securingnode}.
According to~\cite{RFC7519}, these should be attached to the
\texttt{Authorization} part of the HTTPS header, along with a hardcoded
\texttt{Bearer} part, preceding the token as the value of the header.
\todo[inline]{
  Mention this in the security analysis.
}
In LessPM, the JWT claim contained data we could use to validate and
authenticate a user between HTTPS requests.

LessPM utilizes two functionalities of JWTs to ensure the validity of the
Base64-encoded token.

First, all tokens passed back and forth between the LessPM server and client
receives an expiry time.
We enforce this technique to ensure and maintain the security of accessed
resources, requiring a user to reauthenticate when the expiry timestamp is
passed.

Second, The tokens used in LessPM are cryptographically signed with RSA
Signature Scheme with Appendix - Probabilistic Signature Scheme (RSASSA-PSS)
using Secure Hashing Algorithms with 512-bit size (SHA-512), and is derived
from the Rivest-Shamir -Adleman (RSA) encryption.
The encryption is performed with a key-pair stored on the server to mitigate
the risk of token forgery and so LessPM can maintain the validity and
authenticity of the tokens created.
This prevents unknown authorities from constructing or hijacking tokens
belonging to LessPM\@.\footnote{
  Hijacking a token could happen by a man-in-the-middle attack.
  This is done by a third-party individual listening and intercepting traffic
  in order to either read data or input their own in a client's request.
  This would allow an attacker to gain access to privileged information.
}
Other signature algorithms that can be used include Hash-Based Message
Authentication (HMAC), Elliptic-Curve Digital Signature Algorithm (ECDSA), or
a plaintext secret stored on the server.
LessPM uses RSASSA-PSS with SHA-512 due to the asymmetric operations the
algorithms perform on the encryption and decryption, whereas HMAC and ECDSA is
based on symmetric- and elliptic curve cryptography.


\subsubsection{JSON Web Encryption}
By default, JWTs are not encrypted.
It is out of the question to unnecessarily expose any sensitive details about
the LessPM server's inner workings.
This lead to the decision of finding a way to encrypt the contents of the 
token before they are sent from LessPM's server.
Without the encryption, any attacker could decode the Base64 encoding and
gain access to the claim that LessPM uses to authorize and authenticate.

To perform this operation, LessPM seeks inspiration from JWEs and
RFC7516~\cite{rfc7516}.\footnote{
  To properly implement JWE, metadata should be encoded into the token, such as
  what algorithm performs the encryption~\cite{rfc7516}. For further
  explanation, see Section~\ref{sec:futurework}.
}
\todo[inline]{
  Diagram that shows the operation
}
Before the RSASSA-PSS signed JWT is attached to the HTTPS header,
LessPM encrypted the Base64-encoded data using AES-256, before turning them into
a new Base64 encoded string.
Encrypting the token with AES-256 elevated the overall security of the
JWTs, making them more suitable for transferring sensitive data between the
intercommunicating entities in LessPM\@.
As a final step, the ciphertext created by AES-256 is again encoded into a
new Base64.
When LessPM again receives this claim, we used it to gather details about the
user who made the request and responded to the request accordingly.
An example of this encoding can be seen in
Appendix~\ref{subsec:base64-and-ciphertext}.

\todo[inline]{
  Turn this into a diagram, because Jacky love those.
}
The process functions as follows in LessPM:
\begin{enumerate}
  \item The client completed an authentication request.
  \item A claim was constructed and created on the server
  The claim could have been any data the server wished to use to authenticate
  the legitimacy of a future request.
  \item The claim was signed with the desired algorithm.
  This could also have been a secret, stored on the server.
  \item The claim is encoded into a Base64 string.
  \item The Base64 encoded string is passed to AES-256 encryption algorithm,
  creating a ciphertext.
  \item The cipertext is created into a new Base64-encoding, which is the
  final token.
  \item As part of the response to a request, the server appended the
  token.
  \item The client received the token and carried it upon the next performed
  request.
  \item When the server received the token again, it decodes the
  Base64-encoded ciphertext.
  \item The ciphertext is decrypted, decoded and reconstructed to a claim.
  \item The \texttt{exp} property of the JWT is validated and LessPM takes
  action accordingly.
\end{enumerate}

\todo[inline]{
  Put this paragraph somewhere in the WebAuthn section
  Further, the server checked and verified the token before proceeding with any
  requests made to it.
  This information is updated and inspected between each
  re-render~\cite{react-component} of the client.
}

\todo[inline]{
This should go in the new section of WebAuthn in methodology. Somewhere.
For each step in registration~\ref{subsubsec:metho-registration},
authentication~\ref{subsubsec:metho-authentication}, and
password creation/retrieval~\ref{subsubsec:creation-and-retrieval}, different
tokens with different information are constructed, verified, and
discarded.\footnote{
  The token for password creation/retrieval is the same as verifying that a user
  is signed in and authorized to create entries on behalf of the user.
}
We opted for this approach so that an exposed token could not serve as more than
one entry point for one of the steps.
}



\subsection{WebAuthn}\label{subsec:webauthn-methodology}
We used WebAuthn to perform passwordless authentication in LessPM\@.
WebAuthn is a collaborative, open standard between the FIDO Alliance and W3C\@.
The aim of the standard is to implement a secure, robust key-based
authentication system for the web, to strongly authenticate users through
biometrics~\cite{webauthn_level_2}.

The concept relies on the use of a third-party device, called an
Authenticator Device (AD), which leverages asymmetric cryptography.
These devices employ biometric or hardware-based mechanism to provide secure
and reliable means of authenticating a user in LessPM\@.

Upon registering in LessPM, the AD generated a key-pair called a Passkey.
This Passkey contained a CID uniquely generated for each registered
key-pair~\cite{webauthn_credential_id,webauthn_public_key_credential}, per
services registered on the AD\@.

Further, if a malefactor gains access to an individual's Passkey, they might
compromise one specific service, whereas a traditional password could
potentially compromise multiple services where password reuse
occurs~\cite{wang2018next}.
This suggests that WebAuthn could create a stronger level of security, whereas
a traditional password exposed in a leak might leave a user susceptible to
losing access across services and devices through the reused password.
Preventing the potential security threat of password reuse is a top priority
for LessPM\@.
That is why WebAuthn has been chosen as the preferred standard for
passwordless user authentication in LessPM\@.

WebAuthn further attempts to mitigate phishing, man-in-the-middle, and
brute-force attacks through its intuitive design~\cite{webauthn_level_2}.
By leveraging the increasing market of smartphones with
biometrics~\cite{statista-biometric-transactions}, WebAuthn becomes a natural
extension, not just for LessPM, but for general authentication.

In order to increase the security of authenticated requests, each of the
steps using WebAuthn, Registration~\ref{subsubsec:metho-registration},
Authentication~\ref{subsubsec:metho-authentication} and
Password- Creation \& Retrieval~\ref{subsubsec:creation-and-retrieval})receives
different tokens with different information and is constructed, verified, and
discarded separately in LessPM.\footnote{
  The token for password creation/retrieval is the same as verifying that a user
  is signed in and authorized to create entries on behalf of the user.
}
We made this decision to further elevate security in LessPM, so that an
exposed token could not serve as more than one entry point for one of the steps.

\subsubsection{Registration}\label{subsubsec:metho-registration}
When a user tried registering in LessPM, the client performs a
registration request to the server, called a Relying Party (RP), carrying a
relevant User Identifier (UID. i.e.\ username, phone number, email, Etc.).
This is done to check whether a user with a similar UID existed\footnote{
  WebAuthn does not require to check whether the CID generated by the AD exists.
  However, accepting users with similar identifier might leave a risk of
  providing unauthorized access.
  Hence why LessPM performs such a check.
} in the system, and LessPM denies the registration process if it does.

If there are no users with a similar UID, LessPM responds by initiating a
Registration Ceremony (RC), generated a challenge and a Unique User
Identifier (UUID), which serves as the body of the HTTPS response.
LessPM also attached a JWE to the HTTPS response \texttt{Authorization} header
which is sent to the client.

The \texttt{expiration} time for this claim is set to one minute to allow the
user some time to authenticate.\footnote{
  WebAuthn describes a timeout performed within the system. In our case, this
  timeout is one minute.
  However, we chose to add this extra step to secure LessPM further, and
  JWEs require an expiry time (See Section~\ref{subsec:auth-and-auth}).
}
The claim expired after this minute, and the user would then have to restart the
process.
This expiration timer was a decision we made to emphasize security within LessPM
further.

Using the issued challenge, LessPM's client called the browser-integrated
WebAuthn API\footnote{
  In LessPM's case, this is the \textit{navigator.credentials} API provided by
  the browser.
}, prompting the user to utilize their AD to create a new Passkey credential
through the Client To Authenticator Protocol (CTAP2)\footnote{
  The user is prompted to use their AD to prove their presence, which can
  involve facial recognition, providing a fingerprint, or any other modality
  supported by the device that the user chooses.
} for
LessPM\@.
At this point, it is entirely up to the user to decide what device to use to
authenticate.
For our development, we used an iPhone 13 Pro Max, a Samsung S21, and a
Samsung Galaxy A52 to test authentication.\footnote{
  Other alternatives included a YubiKey, NitroKey, Etc.
}
We scanned the QR codes prompted through our phones, which initiated the
key-pair generation on the AD after a successful biometric scan, such as
facial recognition.\footnote{
  There is a question of concern that photography can bypass facial recognition.
  Apple uses built-in sensors to scan depth, colours and a dot projector to
  create a 3D scan of a person's face.
  However, this approach prevents the use of photography to authenticate through
  their FaceID and TrueDepth technology~\cite{apple-support}.
}

Finally, the AD signs the challenge using the private key stored on the
device.
The created public key, signed challenge, and additional metadata are combined
into a public key credential object, which is forwarded to the client
through CTAP2 and then sent to LessPM in a new request.

Before the new request reaches the RP/LessPM, an authentication middleware
checks and validates the JWE, denying the request with an \texttt{Unauthorized}
HTTPS status code if the request is invalid.\footnote{
  Validity in this context means not timed out, tampered with, or similar.
}
If the RP/LessPM can validate the authenticity of the signed challenge and
public key through WebAuthn, the user was stored in the database, along with the
UUID generated.

Thus completing the RC\@.
This process can be seen in Figure~\ref{fig:webauthn}


\subsubsection{Authentication}\label{subsubsec:metho-authentication}
When a user wishes to perform authentication (commonly referred to as
\texttt{logging in}), much of the same procedure occurs in LessPM.
The user issued an authentication request in the LessPM's client, carrying
the UID the user used to register.
The client included this information in the body.

Upon receiving an incoming request, the RP/LessPM checked the database for
the containing UID\@.
LessPM immediately rejected the request, should the user not exist in the
database.
If LessPM found an associated user with the UID in the database, the server
responds by initiating and issuing an Authentication Ceremony (AC),
collecting the public key associated with the user from the database and
generating a new challenge.
Upon validation\footnote{
  Validation in this context only means that the key stored in the database
  is valid.
} of the public key, LessPM creates a new JWE, which, like registration, also
receives an \texttt{expiry} time of one minute for the user to authenticate,
attached to the HTTPS \texttt{Authorization} header.

The challenge is issued and LessPM's client then calls the
browser-integrated WebAuthn API again, prompting the use of the original AD
to validate and sign the challenge through CTAP2.
Unlike the registration process, the AD now yields a signed signature based
on the challenge issued by the RP/LessPM\@, which is transferred back to
LessPM's client through CTAP2 with some AD specific
data~\cite{webauthn_authenticator_data}.
A new HTTPS request is then issued with the signed challenge and the AD specific
data, and the RP/LessPM then validated the signature using the stored public
key.
LessPM considers the user authenticated if the RP accepted and validated the
signed challenge.
If the AC is unsuccessful at any point in the ceremony, the ceremony is
aborted and considered invalid.

Upon success a new JWE is generated.
This time, the \texttt{expiry} time-to-live is set to a 15-minute\footnote{
  The specification does not specify any upper- or lower bounds for the
  expiry time~\cite{RFC7519}
}
timeframe, allowing the user some time to perform necessary activities
within LessPM\@.
The WebAuthn-related process can be seen in Figure~\ref{fig:webauthn} and the
JWE-related process can be seen in Figure~\ref{fig:JWT-process}.

\subsubsection{Password Creation \& Retrieval}\label{subsubsec:creation-and-retrieval}
Passwords are sensitive in nature, so it seems only natural in a security
context to enforce an extra level of authentication upon retrieving and creating
one unique password in LessPM.\footnote{
  We retrieved a complete list of the user's passwords upon successful authentication.
  The hashed version of the password is stripped of the returned values to protect and enforce security.
}

The following options are presented to a user when they initiate a password
creation process in LessPM's client:

\begin{itemize}
  \item \textbf{User Identifier}: An identification that the user wants to
  associate with the password entry they are storing.
  Such as a username, phone number, or email.
  \item \textbf{Website}: A URL or similar where the password belongs.
  \item \textbf{Password}: The user is prompted with the input to create a
  strong password automatically, choosing options such as numbers, special
  symbols, smaller or larger characters, and the length.
  As an option, the user was also permitted to construct their password but
  warned by a warning saying that this option is less secure.
\end{itemize}

As a final step before a password is created and stored, the user is prompted to
reauthenticate with their AD\@.
This is done in a similar manner as described for registration (See
Section~\ref{subsubsec:metho-registration}) and authentication (See
Section~\ref{subsubsec:metho-authentication}).

To retrieve the plaintext version of the password, LessPM enforces a new
authentication request through the AD\@.
We made this decision to attempt to ensure the owner\footnote{
  In this context, we distinguish between \texttt{user} and \texttt{owner}
  as the person that created the password, not the person currently using
  LessPM.
} of
the password's presence\footnote{
  This approach is also used in password managers on phones to avoid situations
  where a user might have left their computer unlocked.
}.
In such a scenario where the owner left their computer open while logged in to
LessPM, a malefactor would not be able to simply retrieve a password at will.
To decrypt a password, LessPM also required the CID from the AD to perform
the key-reconstruction for the encryption algorithm employed in LessPM\@.
Further details about how encryption is achieved can be seen in
Section~\ref{subsec:password-encryption}.

\subsection{Cors}\label{subsec:cors}
Cross-Origin Resource Sharing (CORS) must be configured correctly when the
server and client are running separately on different ports.

When a web page tries to access a resource hosted on another domain, browsers
perform an additional request to the server, called a \texttt{``preflight''}.
The preflight request determines whether the
request that the web page is trying to make to the server is allowed.
This request is done through the \texttt{OPTIONS} method in HTTP and contains
some information about the origin, accepted Content-Type, and similar for the
actual request.

The server responds to this with what methods and headers are allowed, denying
the actual request from ever happening if the preflight is not successful.

We constructed a CORS \texttt{layer}\footnote{
  In this context, we referred to a layer as a wrapper around all other
  requests.
} which contained the two domains for the server and client, permitted
credentials\footnote{
  To pass the JWT token back and forth between the server
} and then permitted the two HTTP methods \texttt{POST} and \texttt{GET}.
We also ensured the \texttt{Content-Type, Authorization}, and \texttt{Cookie}
headers are permitted.

Any other methods or headers should abort the request in the preflight.

\subsection{Cookie}\label{subsec:cookie}
JavaScript can access and manipulate \texttt{Cookies}~\cite{he2019malicious}.
We utilized the browser's local cookie storage to attempt secure authentication
between requests.\footnote{
  The cookie storage in a browser is subject to any vulnerabilities that can be
  performed on an SQLite database while having access to the computer where it
  is running.
}
We attempted a couple of strategies listed below to fortify the cookie that
LessPM set in the browser against a malefactor:
\begin{itemize}
  \item \textbf{Strict SameSite}:
  This ensures that the cookie is safeguarded against Cross-Site Request Forgery
  (CSRF) attacks and remains restricted to its original origin domain.
  \item \textbf{Expires}:
  Once the system authenticated the user, it gave the cookie a Time-to-Live
  (TTL) mechanism similar to the JWE, which remained valid for only 15 minutes.
  \item \textbf{Secure}:
  We applied the \texttt{Secure} attribute to ensure that the cookie was only
  accessible through the HTTPS protocol.
  This protocol encrypts the data being sent back and forth between the client
  and the server, attempting to avoid eavesdroppers.
  \item \textbf{HttpOnly}:
  Setting HttpOnly tells the browser to make this cookie inaccessible through
  JavaScript.
  This property is important to mitigate session hijacking.
\end{itemize}

\subsection{Password Encryption}\label{subsec:password-encryption}
A password can be stored and hashed using multiple vectors to increase
security in a typical authentication scheme.
Such measures include using a \texttt{salt}\footnote{
  Salting is the process of adding a randomly generated string consisting of
  arbitrary characters to the password before creating a hash~\cite{Kharod2015}.
  This randomness of the salt makes identical passwords hash to different
  values, which can then be stored in the database.
} and \texttt{pepper}\footnote{
  A pepper is the process of adding a hardcoded string to the password.
  Unlike the salt, the pepper is often stored in the code and used as an
  extra measure to increase security.
  A sufficently large pepper will require a malefactor to spend extra time to
  compute a hash.
}.
When trying to authenticate, the user would provide their UID and password.
These values are collected from the database, where a password can be stored in
any form from plaintext to a hashed variation through a hashing algorithm
such as PBKDF2, Argon2, or the SHA-family.
After the values are collected from the database, the authentication scheme
checks them against the provided values from the user.
If a \texttt{salt} and \texttt{pepper} is part of the authentication scheme,
they are both required to be appended (in the correct order) to the
user-provided password, before the hashing and validation can take place.
Upon successful validation, the user is considered authenticated and logged in.

Securing the user passwords in LessPM made this process more complicated than
the one described above.
Furthermore, since the passwords in a general password manager should be a
randomly generated string unknown to the applicant, we cannot hash the
user-provided input and compare that hashed value.
This influenced the decision to introduce encryption on the passwords in
LessPM\@.

\subsubsection{Advanced Encryption Standard}
We decided on the Advanced Encryption Standard (AES) with a 256-bit key-size
(AES-256).
AES has, since its inception in 1998, become the gold standard for encrypting
various information across
applications~\cite{schneier2015applied,rijndael_book}.
In 2001, it was adopted as the successor of the leading Data Encryption
standard (DES) by The National Institute of Standards and Technology (NIST)
in 2001~\cite{nist_aes_winner}.
AES operates on fixed-sizes units of data referred to as
\texttt{blocks}~\cite{nistfips197blocks}, supporting keys of sizes 128-, 192-
and 256-bit~\cite{nistfips197intro}.
A Substitution-Permutation Network (SPN) structure forms the basis of the
design, which achieves a high level of encryption and security in LessPM
through multiple rounds of processing by combining substitution and
permutation~\cite{nistfips197specification}.
AES with 256-bit key length (hereafter referred to as AES-256) employs a 256
-bit key and consist of 14 rounds of encryption, offering an advanced level
of security compared to its counterpats with shorter key lengths and fewer
rounds~\cite{nistfips197256}.
In each round of encryption, AES-256 undergoes four primary transformations,
operating on a $4\times4$ block, as described below:
\newpage
  \begin{enumerate}
    \item \textbf{SubBytes} is a non-linear substitution step where each byte is
    replaced with another according to a predefined lookup table.
    \item \textbf{ShiftRows} cyclically shifts each row of the state over a
    certain number of steps.
    of the State over varying numbers of bytes while preserving their original
    values.
    \item \textbf{MixColumns} is a process that works on the columns of the
    state by combining the four bytes in each column through a mixing operation.
    \item \textbf{AddRoundedKey} involves combining a subkey with the state\footnote{
      The term \texttt{state} refers to an intermediate result that changes as
      the algorithm progress through its phases.
    }
    ~by applying a bitwise XOR operation.
  \end{enumerate}
\newline
The larger key-size in AES-256 provides an exponential increase in the number
of possible keys for each password encrypted in LessPM, making it
significantly more resilient to brute-force attacks and further solidifying
its position as a robust encryption standard for safe-guarding sensitive
information.\footnote{
  The practical number of potential keys for an AES-256 implementation is
  $2^{256}$ possibilities. This gives us an approximation of
  $1.1579209 \times 10^{77}$ options.
  The number is theoretical, as this is a worst-case scenario of options an
  attacker must go through to find the right key.
}

\subsubsection{AES-256 in LessPM}
AES-256 requires a 256-bit key to encrypt and decrypt data.
Since the CID generated by WebAuthn is unique and random for each application
and device, this serves as a basis for constructing the key used to encrypt
passwords in LessPM\@.

We based the key for each password on the following premise:
\begin{enumerate}
  \item
  We took advantage of the fact that each CID is unique in every application
  (see Section~\ref{subsec:webauthn-methodology}).
  We therefore used 192-bits of the string that is the CID\@, converting it to
  integers.
  Since every CID is unique depending on the device, some CIDs are smaller
  than 192-bit and some are longer.
  To compensate for this, LessPM constructed and generated a random padding
  to reach the remaining difference when the CID is smaller than 192-bit.
  \item
  We appended each key with a random 128-bit salt of integers and stored these
  bits with the entry in the database.
  \item
  Then we add a 128-bit of pepper collected from an environment variable to
  in LessPM's code to finish the key.
\end{enumerate}
We used these 448 bits as the input for the key-derivation function,
Argon2 (See Section~\ref{subsec:hashing}).
Upon hashing, each key is also subject to a 128-bit salt explicitly generated
for the password's key in LessPM\@.
This process happened individually for all passwords constructed and stored
within LessPM\@.

\subsection{Hashing}\label{subsec:hashing}
\todo[inline]{
When we searched for a good key-derivation function, we first came across
  Password-Based Key Derivation Function 2 (PBKDF2).
We saw PBKDF2 as a good solution for the project, but after researching the
topic further, we ended up with Argon2.

Argon2 is regarded by some to be more secure than PBKDF2 due to its modern
design considerations, including protection against side-channel attacks and
memory-hardness, which make it more resistant to brute-forced and rainbow
table attacks.
Argon2 also offers customization of parameters for flexibility in adapting to
changing hardware technologies and security requirements.
PBKDF2 offers to set an amount of iterations to construct the hash and which
pseudorandom function to use.
}

Since LessPM is required to run in as safe of an environment as possible,
Argon2's configuration option is an excellent solution.

\subsubsection{Configuring Argon2}
Argon2's hashing output is dependable on
configurations~\cite{argon2specs}.\footnote{
  Dependable in this context refers to each configuration that can
  possibly be constructed.
  An instance of Argon2 with 256 Megabytes of \texttt{memory} will not return
  the same hash as 255 Megabytes.
  The same is true for the amount of \texttt{iterations} and
  \texttt{parallelism}.
}
Given that we emphasized security, we opted for the \texttt{Argon2id}
configuration, which gave us equal protection against side-channel and
brute-force attacks.

\todo[inline]{
  A part of Argon2's customizability is the offer to set the option for a
  required amount of memory to do the hashing.
}
These options would force a malefactor to use a specific amount of memory for
each attempt to construct the hash.
The only way a malefactor can get passed this requirement is to purchase more
memory.\footnote{
  As a side note, the increase in memory usage will scale as technology evolves
  and more memory becomes common.
}

For LessPM, we used 128 Megabytes of memory to construct the hash.\footnote{
  According to~\cite{argon2specs}, the reference implementation using Argon2d
  with 4Gb of memory and 8-degree parallelism, the hashing process should take
  0.5 seconds on a CPU with 2Gz. However, we were unsuccessful in seeing
  anywhere close to similar results.
}
We went for the default suggestion of two iterations to complement the memory.
To finalize the configuration, we added 8 degrees of parallelism since the
system where we developed it consists of 8 cores.~\footnote{
  The degree of parallelism is affected by how many cores a CPU
  contains~\cite{argon2specs}.
}

\todo[inline]{The following section}

\subsection{Security Analysis}\label{subsec:security-analysis}
The security of a passwordless password manager is of paramount importance to
protect sensitive user infromation and prevent unauthorized access.
In this subsection, we conduct a comprehensive security analysis, identifying
  potential attack vectors and outlining the defensive measures that have
  been implemented to mitigate tehse risks.

The defensive measures include industry-standard security protocols such as
  HTTPS, JWT/JWE, Argon2 hashing, salt and pepper techniques, CORS
  configuration, and secure cookie management.
Additionally, different JWTs are used for registration, authentication, and
  password creation/retrieval to minimize the risk of exposure.
These defensive measures collectively aim to provide a robust and secure
  architecture for LessPM, safeguarding against various threats and ensuring
  the confidentiality, integrity, and availability of user data.
\begin{enumerate}[label=$\blacktriangleright$]
  \item \textbf{HTTP}

  \textbf{Weakness:}

  Any server running HTTP (Hypertext Transfer Protocol) is passing data
  between a server and a connecting client.
  With HTTP, this data is unencrypted.
  A malefactor can eavesdrop this data, performing a man-in-the-middle attack,
  replacing information, reading tokens, perform header inections, Etc. with 
  this unencrypted traffic.

  \textbf{Defensive Mechanism:}

  Upgrading the HTTP connection to HTTPS ensures that communication between the
  client and server is encrypted through Transport Layer Security (TLS).
  It provides confidentiality and integrity of data transmitted over the
  network, making it more difficult for attackers to intercept or tamper with
  sensitive information.

  \item \textbf{JSON Web Tokens (JWT)}

  \textbf{Weakness:}

  JWTs are signed with a secret specific to the server.
  However, if this secret is discovered or leaked, a malefactor could use this
  information to sign their own tokens and provide their own information.
  Since JWTs are signed and not encrypted, any malefactor who receives access
  to a token can decode the Base64 format and read the tokens information in
  clear text.

  \textbf{Defensive Mechanism:}

  Properly signining the tokens with a strong algorithm prevents information
  from being tampered with.
  An attacker can still read the information in clear-text but can not forge
  their own tokens.
  LessPM uses RSASSA-PSS with Sha-512 to sign and encode tokens, decoding
  them with the keypair when received.
  This ensures integrity and authenticity of the JWT, as the signature is
  verified by the server.
  RSASSA-PSS is a robust and secure signature algorithm that provides protection
  against various attacks, such as collision attacks and length extension
  attacks.
  To further impose a level of security, LessPM takes advantage of different
  tokens for registration, authentication, and authorization.

  \item \textbf{JSON Web Encryption (JWE}

  \textbf{Weakness:}

  JSON Web Encryption are JWTs signed and encrypted using an algorithm for
  signing and encryption.
  However, if this encryption algorithm is not potent or weak (such as a
  Caesar Cipher), an attacker can break the encryption or even encrypt their own
  tokens.
  JWEs are also subject to scenarios such as lack of key rotation, insecure
  length of the encryption key and improper implementation (See
  Section~\ref{sec:futurework}).

  \textbf{Defensive Mechanism:}

  To combat these weaknesses, LessPM implements encryption partly inspired by
  the JWE standard.
  LessPM encrypts the token using AES-256, a strong encryption scheme capable
  of $2^{256}$ different keys.
  This ensures that the data is confidentially kept and protected from
  unauthorized access.
  While inspired, it is important to note that JWEs have not been properly
  implemented in LessPM (See Section~\ref{sec:futurework}).

  \item \textbf{Hashing Passwords \& Keys}

  \textbf{Weakness:}

  Hashing is a one-way process to convert information (such as passwords)
  into a fixed-size string of characters, typically a fixed-length hash value.
  However, the hashing process is susceptible to a malefactor using
  precomputed hash-values for large number of possible passwords, stored in a
  lookup table called a Rainbow table attack.
  Hashes are also vulnerable to dictionary attacks, and not being properly
  implemented without using a salt and preferable a pepper, or brute-force
  attacks.

  \textbf{Defensive Mechanism:}

  LessPM takes advantage of the latest within public hashing functionality,
  through Argon2.
  Argon2 constructs a memory-hard and computationally expensive hash that
  provides protection against brute-force, rainbow table, dictionary and
  side-channel attacks by requiring a significant amount of computational
  resources and time to compute the hash.
  Finally, LessPM takes advantage of using both a random salt unique for each
  password and a pepper stored in the source code of 128-bit for both.
  This creates a higher threshold for a malefactor by requiring access to the
  database and the source code to be able to quickly compute a hash.
  The salt and pepper are added to the key of the AES-256

  \item \textbf{Password Authentication}

  \textbf{Weakness:}

  A password is something a secret that a user knows.
  Secrets belonging to an individual will always be potentially accessible to
  a malefactor.
  There are multiple vectors that can be used, such as a user being careless
  and writing the password down on a piece of paper or storing the secret in an
  unencrypted file.
  Further, a password is required to be stored somewhere for a server to
  authenticate the user, preferably with an email for recovery.
  This creates two new vectors for a malefactor to exploit, the persisted
  storage on the server or the recovery process.

  \textbf{Defensive Mechanism:}

  LessPM takes advantage of WebAuthn to avoid the usage of passwords to
  authenticate the user.
  The user is required to have two vectors for authentication; their
  authenticator device and the biometrics, which is the user itself.
  It is, however, important to mention that any authenticator device is
  vulnerable to being compromised.
  Should a device be compromised or infected with malicious software (Malware),
  it could be used to intercept or manipulate authentication requests.

  \item \textbf{Storing passwords in plaintext}

  \textbf{Weakness:}

  Storing a password can be done in plaintext in a database protected by a
  password.
  However, should an attacker get access to the database, all passwords and
  their plaintext are compromised.
  Storing the passwords in plaintext makes the information accessible to any
  individual with access to the database, which causes a privacy risk.

  \textbf{Defensive Mechanism:}

  LessPM encrypt all passwords with one of the strongest public symmetric
  encryption processes, AES-256.
  Each key for each password is unique, and consists of the 192-bit CID from
  WebAuthn generated when the user registered (See
  Section~\ref{subsec:password-encryption}).
  This CID is unique to each service the user is registered with.
  Along the 192-bit CID, 128-bit is used from a salt and 128-bit used from a
  pepper.
  The salt is stored with the entry in the database.
  Using both salt and pepper enhances the security of the derived key.

  \item \textbf{
    Cross-Site Scripting,
    Cross-Site Request Forgery
    \& Unauthorized Access}

  \textbf{Weakness:}

  Any HTTP server improperly configured is subject Cross-Site Scripting (XSS)
  and CSRF\@.
  XSS is a process where a malefactor could inject a malicious script from a
  different origin, and execute said script in the context of a user's browser.
  This could lead to a situation where a malefactor could get access tokens
  or cookies, or interact with the webpage a user is viewing through the
  script.
  CSRF is a type of security vulnerability where a malefactor tricks a user
  to unknowingly make unauthorized requests on a trusted website, such as a
  bank or similar.
  This can lead to actions being performed on a user's behalf unintentionally,
  without the user's consent, leading to unauthorized access.

  \textbf{Defensive Mechanism:}

  CORS is a measure that enforces a strict policy for which domains and services
  are permitted to access certain resources on the server.
  LessPM takes advantage of CORS by permitting the server itself and the
  client associated domain to access resources on the server.
  This is a preventive measure put in place to allow the communication
  between the client and the serve, even though they are running on different
  ports, and potentially different domains.
  This ensures that only authorized clients can access the server's resources,
  preventing unauthorized cross-origin requests and protecting against
  cross-site scripting (XSS) and cross-site request forgery (CSRF) attacks.

  \item \textbf{Cookies}

  \textbf{Weakness:}

  A cookie is vulnerable to many aspects of security.
  Improperly stored, a cookie can be accessed through JavaScript, sent to
  different domains, get hijacked, excessive expire time, or even poisoned.\footnote{
    Cookie Poisoning is the process where an attacker modifies the content of
    the cookie to inject malicious data or bypass security controls.
  }

  \textbf{Defensive Mechanism:}

  LessPM only uses one cookie, which contains the encrypted JWT\@.
  The cookie is protected through built-in browser-features such as
  restricted to the same origin that the cookie came from, and cannot be sent
  anywhere else, further preventing XSS of sensitive information\@.
  The cookie is expired after 15 minutes, which is extensive amount of time
  for a user to have authorized access to their passwords, before the need to
  reauthenticate their identity.
  Upon creation, LessPM makes sure that the cookie becomes set to secure.
  This prevents the cookie from being sent over an insecure HTTP connection,
  limiting it to HTTPS\@.
  Finally, the cookie is HttpOnly, so that the cookie can't be access through
  JavaScript, reducing the attack vector of SS further.

\end{enumerate}

%\todo[inline]{The preceeding section.}