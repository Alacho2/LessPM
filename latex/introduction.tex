In today's digital landscape, utilizing various identifiers (such as usernames,
email addresses, or phone numbers) combined with passwords has become a
prevalent method for verifying an individual's identity and ensuring their
authorization to access restricted materials.

This is not a novel approach.
Polybius' \textit{The Histories}~\cite{perseus_tufts} contains the first
documented use of passwords, describing how the Romans employed
``\textit{watchwords}'' to verify identities within the military.
This provided a transparent, simple way to allow or deny entry to restricted
areas of authorized personnel only.
The story of secret writing (in this context referenced as cryptography) goes
back the past 3000 years~\cite{history_cryptography_cryptanalysis}, where the
need to protect and preserve privacy between two or more individuals blossomed.

Fernando J. Corbató is widely credited as the father of the first
computer password when he was responsible for the Compatible Time-Sharing
System (CTSS) in 1961 at MIT~\cite{levy1984hackers}.
The system had a \texttt{"LOGIN"} command, which, when the user followed it by
typing \texttt{"PASSWORD"}, had its printing mechanism turned off to offer
the applicant privacy while typing the password~\cite{ctss_programmers_guide}.
Given the long history of passwords and their importance, one could argue that
it was a natural and judicious step in the evolution of computer systems.

The study \textif{"The Memorability and Security of Passwords"}, conducted in
2004, provides insights into password creation strategies, including tips on
improving password entropy and methods for easy recall of passwords~\cite{
    yan2000password}.
With an emphasis on diversity in character selection, password length, and
avoiding dictionary words, the study suggests that acronym-based passwords offer
a delicate balance between memorability and security~\cite{yan2000password}.

However, as technology has advanced, the limitations of password-based
authentication have become increasingly apparent, leading to the development of
more sophisticated methods like Universal Authentication Framework
(UAF)~\cite{fido_uaf_overview} and WebAuthn~\cite{webauthn_level_2} through the
Fast IDentity Online Alliance \href{https://fidoalliance.org}{(FIDO)} and
The \href{https://www.w3.org}{World Wide Web Consortium} (W3C).

\newcommand{\assymetricCrypto}{\footnote{Asymmetric cryptography uses a
key-pair consisting of public and private keys. The public key encrypts data,
while the private key decrypts it. The keys are mathematically related, but
    deriving one from the other is infeasible, ensuring secure communication and
    data exchange.}}

This report explores the implementation of LessPM, a passwordless password
manager that leverages WebAuthn (Web Authentication), an open standard and
collaborative development effort between FIDO and W3C to provide a secure
authentication experience through biometric scanners on devices such as a
smartphone or a hardware authenticator device to provide a secure experience.
Free from the constraints of traditional passwords, while placing a strong
emphasis on security.
\todo[inline]{
    LessPM is designed with a multi-layer security approach to ensure
    confidentiality and integrity of user authentication and the passwords
    belonging to other services that user's are registered on.
    By utilizing WebAuthn's asymmetric
    cryptographic~\assymetricCrypto nature~\cite{webauthn-2-registering},
    LessPM authenticates users strongly through the help of authenticator
    devices such as a smartphone or hardware authentication device in possession
    of the user.
    When registering in LessPM, the user is prompted to use a biometric
    sensor on their device. The device constructs a keypair that the
    implemented WebAuthn server receives and stores the public key of the
    keypair before registering the user.
    This process can be seen in Figure~\ref{fig:registration}.

    When the user attempts to authenticate in the future, the user is again
    prompted to use the biometric sensors on their device. Upon initiating this
    process, the server issues a challenge which is signed by the private key on
    the authenticator device, yielding a signature back when a biometric
    sensor is successful in authenticating the authenticity of the user.
    The server can then verify this signature with the help of the public key
    stored int he previous step to authenticate the user.
    This process can be seen in Figure~\ref{fig:authentication}.

    The aforementioned described processes constructs an environment such that a
    malefactor is required to first access the authenticator device and then
    to bypass the device's biometric sensors security in order to authenticate
    as the user.

    Passwords are encrypted using AES-256, a widely recognized symmetric
    encryption algorithm~\cite{schneier2000secrets,rijndael_book}, with a
    Credential ID (CID) that is derived from the public key belonging to the
    keypair from WebAuthn, a 128-bit randomly generated salt, unique for each
    password, and a 128-bit pepper.
}
By examining recent advancements in authentication mechanisms and the related
innovative potential of WebAuthn, we hope to illuminate the prospects of a
passwordless future in digital security.

The findings in this report is not an attempt of getting rid of trivial
concepts such as session hijacking, nor is it an empirical study of user's
perception of the technology seen.