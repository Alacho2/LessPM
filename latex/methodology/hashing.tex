
When we searched for a good key-derivation function, we first came across
Password-Based Key Derivation Function 2 (PBKDF2).
We saw PBKDF2 as a good solution for the project, but after researching the
topic further, we ended up with Argon2.

Argon2 is regarded by some to be more secure than PBKDF2 due to its modern
design considerations, including memory-hardness and protection against
side-channel attacks, which makes it more resistant to brute-force and rainbow
table attacks.
PBKDF2 offers to set an amount of iterations to construct the hash and which
pseudorandom function to use.

\subsubsection{Argon2}
Argon2 aims to provide a highly customizable function, tailored to the needs
of distinct contexts~\cite{argon2specs}.
Additionally, the design offers resistance to both time-memory trade-off
and side-channel attacks as a memory-hard function~\cite{argon2specs}.

The Key-Derivation Function (KDF) fills large memory blocks with pseudorandom
data derived from the input parameters, such as password and salt.\footnote{
  Argon2's intension is to have a 128-bit salt for all applications but this
  can be sliced in half, if storage is a concern~\cite{argon2specs}.
}
The algorithm then proceeds to process these blocks non-linearly for a specific
number of iterations~\cite{argon2specs}.

The KDF offers three configurations, depending on the environment where
the function will run and what the risk and threat models are, which we take
advantage of in LessPM\@.
These can be seen in Figure~\ref{fig:argon2conf}.

\begin{figure}[htbp]
  \centering
  \begin{itemize}
    \item \textbf{Argon2d} is a faster configuration and uses data-depending
    memory access.
    This makes it suitable for cryptocurrencies and applications with little to
    no threat of side-channel timing attacks.\protect\footnotemark
    \item \textbf{Argon2i} uses data-independent memory access.
    This configuration is more suitable for password hashing and key-derivation
    functions.\protect\footnotemark
    ~The configuration is slower due to making more passes over the memory as
    the hashing progresses.
    \item \textbf{Argon2id} is a combination of the two, beginning with
    data-dependent memory access before transitioning to data-independent
    memory access after progressing halfway through the process.
  \end{itemize}
  \caption{The three configurations offered by Argon2~\cite{argon2specs}.}
  \label{fig:argon2conf}
\end{figure}

\footnotetext{
  Side-channel timing attacks analyze execution time variations in cryptographic
  systems to reveal confidential data, exploiting differences in time caused by
  varying inputs, branching conditions, or memory access patterns.
}
\footnotetext{
  Due to the nature of prioritizing security, LessPM uses the third
  configuration.
}

Since LessPM is required to run in as safe of an environment as possible,
Argon2's configurations offers a suitable solution.

Argon2, as a memory-intensive hashing function, demands substantial
computational resources from attackers attempting
dictionary-\footnote{
  A dictionary attack is an approach where an attacker tries to find a hash by
  searching through a dictionary of pre-computed hashes or generating hashes
  based on a dictionary commonly used by individuals or businesses.
} or rainbow table attacks.
This characteristic significantly hampers the feasibility of cracking
passwords using such attacks, constructing an ideal scenario for LessPM's
password vault.
The algorithm's customizability allowed us to adjust its behaviour
based on memory, parallelism, and iterations, catering to LessPM's security
requirements and performance needs.
All of these benefits contributed to why LessPM uses Argon2 instead of PBKDF2.

\subsubsection{Configuring Argon2 for LessPM}
Argon2's hashing output is dependable on
configurations~\cite{argon2specs}.\footnote{
  Dependable in this context refers to each configuration that can
  possibly be constructed.
  An instance of Argon2 with 256 Megabytes of \texttt{memory} will not return
  the same hash as 255 Megabytes.
  The same is true for the amount of \texttt{iterations} and
  \texttt{parallelism}.
}
Given that we emphasized security, we opted for the \texttt{Argon2id}
configuration, which gave us equal protection against side-channel and
brute-force attacks.

With Argon2's customizable offer to set the option for required amount of
memory, iterations, and parallelism, the memory option forces a malefactor to
use a specific amount of memory for each attempt to construct the hash.
The only way a malefactor can get past this requirement is to purchase more
memory.\footnote{
  As a side note, the increase in memory usage will scale as technology evolves
  and more memory becomes common.
}

For LessPM, we used 128 Megabytes of memory to construct the hash.\footnote{
  According to~\cite{argon2specs}, the reference implementation using Argon2d
  with 4Gb of memory and 8-degree parallelism, the hashing process should take
  0.5 seconds on a CPU with 2Gz. However, we were unsuccessful in seeing
  anywhere close to similar results.
}
We went for the default suggestion of two iterations to complement the memory.
To finalize the configuration, we added 8 degrees of parallelism since the
system where we developed it consists of 8 cores.~\footnote{
  The degree of parallelism is affected by how many cores a CPU
  contains~\cite{argon2specs}.
}

As these configurations are crucial for computing the original hash, LessPM's
key hashing provides robust resilience against brute-force and side-channel
attacks~\cite{argon2specs}.

In 2015, Argon2 won the Password Hashing
Competition~\cite{passwordhashing}.\footnote{
  NIST's competition to find an encryption algorithm inspired the Password
  Hashing Competition, but it took place without NIST's endorsement.
}